\clearemptydoublepage
\chapter{Theory}
\label{cha:theory}

\section{Tensor Analysis}
\subsection{Base vectors}

Now, if we have a vector, $\vect{v}$ we can write
\begin{equation}
  \vect{v}=v^{i}\vect{g}_{i}
\end{equation}
where $v^{i}$ are the components of the contravariant vector, and
$\vect{g}_{i}$ are the covariant base vectors.

Similarly, the vector $\vect{v}$ can also be written as 
\begin{equation}
  \vect{v}=v_{i}\vect{g}^{i}
\end{equation}
where $v_{i}$ are the components of the covariant vector, and
$\vect{g}^{i}$ are the contravariant base vectors. 

We now note that
\begin{equation}
  \vect{v}=v^{i}\vect{g}_{i}=v^{i}\sqrt{g_{ii}}\hat{\vect{g}_{i}}
\end{equation}
where $v^{i}\sqrt{g_{ii}}$ are the physical components of the vector and
$\hat{\vect{g}_{i}}$ are the unit vectors given by
\begin{equation}
  \hat{\vect{g}_{i}}=\dfrac{\vect{g}_{i}}{\sqrt{g_{ii}}}
\end{equation}

\subsection{Metric Tensors}
\label{sec:metric tensors}

Metric tensors are the inner product of base vectors. If $\vect{g}_{i}$ are the
covariant base vectors then the covariant metric tensor is given by
\begin{equation}
  g_{ij}=\dotprod{\vect{g}_{i}}{\vect{g}_{j}}
\end{equation}

Similarily if $\vect{g}^{i}$ are the contravariant base vectors then the
contravariant metric tensor is given by 
\begin{equation}
  g^{ij}=\dotprod{\vect{g}^{i}}{\vect{g}^{j}}
\end{equation}

We can also form a mixed metric tensor from the dot product of a contravariant
and a covariant base vector \ie
\begin{equation}
  g^{i}_{.j}=\dotprod{\vect{g}^{i}}{\vect{g}_{j}}
\end{equation}
and 
\begin{equation}
  g_{i}^{.j}=\dotprod{\vect{g}_{i}}{\vect{g}^{j}}
\end{equation}

Note that for mixed tensors the ``.'' indicates the order of the index \ie
$g^{i}_{.j}$ indicates that the first index is contravariant and the second
index is covariant whereas $g_{i}^{.j}$ indicates that the first index is
covariant and the second index is contravariant.

If the base vectors are all mutually orthogonal and constant then
$\vect{g}_{i}=\vect{g}^{i}$ and $g_{ij}=g^{ij}$.

The metric tensors generalise (Euclidean) distance \ie
\begin{equation}
  ds^{2}=g_{ij}dx^{i}dx^{j}
\end{equation}

Note that multiplying by the covariant metric tensor lowers indices \ie
\begin{equation}
  \begin{split}
    \vect{A}_{i} &= g_{ij}\vect{A}^{j} \\
    A_{ij} &= g_{ik}g_{jl}A^{kl} = g_{jk}A_{i}^{.k} = g_{ik}A^{k}_{.j} 
  \end{split}
\end{equation}
and that multiplying by the contravariant metric tensor raises indices \ie
\begin{equation}
  \begin{split}
  \vect{A}^{i} &=  g^{ij}\vect{A}_{j} \\
   A^{ij} &= g^{ik}g^{jl}A_{kl} = g^{ik}A_{k}^{.j} = g^{jk}A^{i}_{.k}
  \end{split}
\end{equation}
and for the mixed tensors
\begin{equation}
  \begin{split}
  A_{i}^{.j} &= g^{jk}A_{ik} = g_{ik}A^{kj} \\
  A^{i}_{.j} &= g^{ik}A_{kj} = g_{jk}A^{ik} \\
  \end{split}
\end{equation}

\subsection{Transformations}

The transformation rules for tensors in going from a $\vect{\nu}$ coordinate
system to a $\vect{\xi}$ coordinate system are as follows: 


For a covariant vector (a rank (0,1) tensor)
\begin{equation}
  {\tilde{a}}_{i}=\delby{\nu^{a}}{\xi^{i}}a_{a}
\end{equation}

For a contravariant vector (a rank (1,0) tensor)
\begin{equation}
  {\tilde{a}}^{i}=\delby{\xi^{i}}{\nu^{a}}a^{a}
\end{equation}

For a covariant tensor (a rank (0,2) tensor)
\begin{equation}
  {\tilde{A}}_{ij}=\delby{\nu^{a}}{\xi^{i}}\delby{\nu^{b}}{\xi^{j}}A_{ab} 
\end{equation}

For a contravariant tensor (a rank (2,0) tensor)
\begin{equation}
  {\tilde{A}}^{ij}=\delby{\xi^{i}}{\nu^{a}}\delby{\xi^{j}}{\nu^{b}}A^{ab}
\end{equation}

and for Mixed tensors (rank (1,1) tensors)
\begin{equation}
  {\tilde{A}}^{i}_{.j}=\delby{\xi^{i}}{\nu^{a}}\delby{\nu^{b}}{\xi^{j}}A^{a}_{.b}
\end{equation}
and
\begin{equation}
  {\tilde{A}}_{i}^{.j}=\delby{\nu^{a}}{\xi^{i}}\delby{\xi^{j}}{\nu^{b}}A_{a}^{.b}
\end{equation}

\subsection{Derivatives}
\label{subsec:function derivatives}

\subsubsection{Scalars}

We note that a scalar quantity $\fnof{u}{\vect{\xi}}$ has derivatives
\begin{equation}
  \delby{u}{\xi^{i}}=\partialderiv{u}{i}
\end{equation}

Or more formally, the covariant derivative ($\covarderiv{\cdot}{\cdot}$) of a
rank 0 tensor $u$ is
\begin{equation}
  \covarderiv{u}{i}=\delby{u}{\xi^{i}}=\partialderiv{u}{i}
\end{equation}

\subsubsection{Vectors}

The derivatives of a vector $\vect{v}$ are given by
\begin{equation}
  \begin{split}
    \delby{\vect{v}}{\xi^{i}} &=
    \delby{}{\xi^{i}}\pbrac{v^{k}\vect{g}_{k}} \\
    &= \delby{v^{k}}{\xi^{i}}\vect{g}_{k}+v^{k}\delby{\vect{g}_{k}}{\xi^{i}} \\
    &= \partialderiv{v^{k}}{i}\vect{g}_{k}+v^{k}\partialderiv{\vect{g}_{k}}{i}
  \end{split}
\end{equation}

Now introducing the notation
\begin{equation}
  \christoffelsecond{i}{j}{k} = \dotprod{\vect{g}^{i}}{\delby{\vect{g}_{j}}{x^{k}}}
\end{equation}
where $\christoffelsecond{i}{j}{k}$ are the Christoffel symbols of the second
kind. 

Note that the Christoffel symbols of the first kind are given by
\begin{equation}
  \christoffelfirst{i}{j}{k} = \dotprod{\vect{g}_{i}}{\delby{\vect{g}_{j}}{x^{k}}}
\end{equation}

Note that
\begin{equation}
  \begin{split}
    \christoffel{i}{j}{k} &= \dotprod{\vect{g}^{i}}{\partialderiv{\vect{g}_{j}}{k}} \\
    &=\dotprod{\vect{g}^{i}}{\christoffelsecond{l}{j}{k}\vect{g}_{l}} \\
    &= \christoffel{i}{j}{l}g^{i}_{.l} 
  \end{split}
\end{equation}

The Christoffel symbols of the first kind are also given by
\begin{equation}
  \christoffelfirst{i}{j}{k}=\frac{1}{2}\pbrac{\delby{g_{ij}}{\xi^{k}}+\delby{g_{ik}}{\xi^{j}}-\delby{g_{jk}}{\xi^{i}}}
\end{equation}
and that Christoffel symbols of the second kind are given by
\begin{equation}
  \begin{split}
    \christoffelsecond{i}{j}{k} &= g^{il}\christoffelfirst{l}{j}{k} \\
    &= \frac{1}{2}g^{il}\pbrac{\delby{g_{lj}}{\xi^{k}}+\delby{g_{lk}}{\xi^{j}}-\delby{g_{jk}}{\xi^{l}}} 
  \end{split}
\end{equation}

Note that Christoffel symbols are not tensors and the have the following
transformation laws from $\vect{\nu}$ to $\vect{\xi}$ coordinates
\begin{align}
  \christoffelfirst{i}{j}{k} &=
  \christoffelfirst{a}{b}{c}\delby{\nu^{b}}{\xi^{j}}\delby{\nu^{c}}{\xi^{k}}\delby{\nu^{a}}{\xi^{i}}+
  g_{ab}\delby{\nu^{c}}{\xi^{i}}\deltwoby{\nu^{c}}{\xi^{j}}{\xi^{k}} \\
  \christoffelsecond{i}{j}{k} &= \christoffelsecond{a}{b}{c}\delby{\xi^{i}}{\nu^{a}}\delby{\nu^{b}}{\xi^{k}}\delby{\nu^{c}}{\xi^{j}}+
  \delby{\xi^{i}}{\nu^{a}}\deltwoby{\nu^{a}}{\xi^{j}}{\xi^{k}} \\
\end{align}

We can now write (BELOW SEEMS WRONG - CHECK)
\begin{equation}
  \begin{split}
    \partialderiv{\vect{v}}{i}&=\partialderiv{v^{k}}{i}\vect{g}_{k}+\christoffel{k}{i}{j}v^{j}\vect{g}_{j}\\
    &=\partialderiv{v^{k}}{i}\vect{g}_{k}+\christoffel{j}{i}{k}v^{k}\vect{g}_{k}\\
    &=\pbrac{\partialderiv{v^{k}}{i}+\christoffel{j}{i}{k}v^{k}}\vect{g}_{k}\\
    &=\covarderiv{v^{k}}{i}\vect{g}_{k}
  \end{split}
\end{equation}
where $\covarderiv{v^{k}}{i}$ is the covariant derivative of $v^{k}$ . 

The covariant derivative of a contravariant (rank (0,1)) tensor $v^{k}$ is
\begin{equation}
  \covarderiv{v^{k}}{i} =\partialderiv{v^{k}}{i}+\christoffel{k}{i}{j}v^{j}
\end{equation}
and the covariant derivative of a covariant tensor  (rank (1,0)) $v_{k}$ is
\begin{equation}
  \covarderiv{v_{k}}{i} =\partialderiv{v_{k}}{i}-\christoffel{j}{k}{i}v_{j}
\end{equation}

\subsubsection{Tensors}

The covariant derivative of a contravariant (rank (0,2)) tensor $W^{mn}$ is
\begin{equation}
  \covarderiv{W^{mn}}{i}=\partialderiv{W^{mn}}{i}+\christoffel{m}{j}{i}W^{jn}+\christoffel{n}{j}{i}W^{mj}
\end{equation}
and the covariant derivative of a covariant (rank (2,0)) tensor $W_{mn}$ is
\begin{equation}
  \covarderiv{W_{mn}}{i}=\partialderiv{W_{mn}}{i}-\christoffel{j}{m}{i}W_{jn}-\christoffel{j}{n}{i}W_{mj}
\end{equation}
and the covariant derivative of a mixed (rank (1,1)) tensor $W^{m}_{.n}$ is
\begin{equation}
  \covarderiv{W^{m}_{.n}}{i}=\partialderiv{W^{m}_{.n}}{i}+\christoffel{m}{j}{i}W^{j}_{.n}-\christoffel{j}{n}{i}W^{m}_{.j}
\end{equation}

\subsection{Common Operators}

For tensor equations to hold in any coordinate system the equations must
involve tensor quantities \ie covariant derivatives rather than partial derivatives.

\subsubsection{Gradient}

As the covariant derivative of a scalar is just the partial derivative the
gradient of a scalar function $\phi$ using covariant derivatives is
\begin{equation}
  \text{grad } \phi = \gradient{\phi}=\covarderiv{\phi}{i}\vect{g}^{i}=\partialderiv{\phi}{i}\vect{g}^{i}
\end{equation}
and
\begin{equation}
  \gradient{\phi}=\partialderiv{\phi}{i}\vect{g}^{i}=\partialderiv{\phi}{i}g^{ij}\vect{g}_{j}
\end{equation}

\subsubsection{Divergence}

The divergence of a vector using covariant derivatives is
\begin{equation}
  \text{div } \vect{\phi} = \diverg{\vect{\phi}}=\covarderiv{\phi^{i}}{i}=\frac{1}{\sqrt{\abs{g}}}\partialderiv{\pbrac{\sqrt{\abs{g}}\phi^{i}}}{i}
\end{equation}
where $g$ is the determinant of the covariant metric tensor $g_{ij}$.

\subsubsection{Curl}

The curl of a vector using covariant derivatives is
\begin{equation}
  \text{curl } \vect{\phi} = \curl{\vect{\phi}}=\frac{1}{\sqrt{g}}\pbrac{\covarderiv{\phi_{j}}{i}-\covarderiv{\phi_{i}}{j}}\vect{g}_{k}
\end{equation}
where $g$ is the determinant of the covariant metric tensor $g_{ij}$.

\subsubsection{Laplacian}

The Laplacian of a scalar using covariant derivatives is
\begin{equation}
  \laplacian{\phi}=\text{div}\pbrac{\text{grad }\phi}=\diverg{\gradient{\phi}}=\mixedderiv{\phi}{i}{i}=\frac{1}{\sqrt{g}}\partialderiv{\pbrac{\sqrt{g}g^{ij}\partialderiv{\phi}{j}}}{i}
\end{equation}
where $g$ is the determinant of the covariant metric tensor $g_{ij}$.

The Laplacian of a vector using covariant derivatives is
\begin{equation}
  \laplacian{\vect{\phi}}=\text{grad }\pbrac{\text{div }\vect{\phi}}-\text{curl } \pbrac{\text{curl }\vect{\phi}}==\mixedderiv{\vect{\phi}}{i}{i}
\end{equation}

The Laplacian of a contravariant (rank (0,1)) tensor $\phi^{k}$ is
\begin{equation}
  \laplacian{\vect{\phi}}=\pbrac{\laplacian{\phi_{k}}-2g^{ij}\christoffel{K}{j}{H}\delby{\phi^{h}}{x^{i}}+\phi^{h}\delby{g^{ij}\christoffel{K}{i}{j}}{x^{h}}}\vect{e}^{k}
\end{equation}
and the covariant derivative of a covariant tensor  (rank (1,0)) $\phi_{k}$ is
\begin{equation}
  \laplacian{\vect{\phi}}=\pbrac{\laplacian{\phi_{k}}-2g^{ij}\christoffel{h}{j}{k}\delby{\phi_{h}}{x^{i}}+\phi_{h}g^{ij}\delby{\christoffel{h}{i}{j}}{x^{i}}}\vect{e}_{k}
\end{equation}

\subsection{Coordinate Systems}
\label{sec:coordinate systems}

\subsubsection{Rectangular Cartesian}

The base vectors with respect to the global coordinate system are
\begin{equation}
  \vect{g}_{i}=\begin{bmatrix} 
    \vect{i}_{1} \\ 
    \vect{i}_{2} \\
    \vect{i}_{3} 
  \end{bmatrix}
\end{equation}

The covariant metric tensor is
\begin{equation}
  g_{ij}=\begin{bmatrix}
    1 & 0 & 0 \\
    0 & 1 & 0 \\
    0 & 0 & 1
  \end{bmatrix}
\end{equation}
and the contravariant metric tensor is
\begin{equation}
  g^{ij}=\begin{bmatrix}
    1 & 0 & 0 \\
    0 & 1 & 0 \\
    0 & 0 & 1
  \end{bmatrix}
\end{equation}

The Christoffel symbols of the second kind are all zero.

\subsubsection{Cylindrical Polar}

The global coordinates  $\pbrac{x,y,z}$ with respect to the cylindrical polar
coordinates $\pbrac{r,\theta,z}$ are defined by
\begin{equation}
  \begin{aligned}
    x = r\cos\theta  & \qquad r \ge0 \\
    y = r\sin\theta & \qquad 0 \le\theta\le2\pi \\
    z = z          & \qquad -\infty < z < \infty
  \end{aligned}
\end{equation}

The base vectors with respect to the global coordinate system are
\begin{equation}
  \vect{g}_{i}=\begin{bmatrix} 
    \cos\theta\vect{i}_{1} + \sin\theta\vect{i}_{2} \\ 
    -r\sin\theta\vect{i}_{1}+ r\cos\theta\vect{i}_{2} \\
    \vect{i}_{3} 
  \end{bmatrix}
\end{equation}

The covariant metric tensor is
\begin{equation}
  g_{ij}=\begin{bmatrix}
    1 & 0 & 0 \\
    0 & r^{2} & 0 \\
    0 & 0 & 1
  \end{bmatrix}
\end{equation}
and the contravariant metric tensor is
\begin{equation}
  g^{ij}=\begin{bmatrix}
    1 & 0 & 0 \\
    0 & \frac{1}{r^{2}} & 0 \\
    0 & 0 & 1
  \end{bmatrix}
\end{equation}

The Christoffell symbols of the second kind are
\begin{align}
  \christoffelsecond{r}{\theta}{\theta}&=-r \\
  \christoffelsecond{\theta}{r}{\theta}=\christoffelsecond{\theta}{\theta}{r}&=\frac{1}{r}
\end{align}
with all other Christoffell symbols zero.

\subsubsection{Spherical Polar}

The global coordinates $\pbrac{x,y,z}$ with respect to the cylindrical polar
coordinates $\pbrac{r,\theta,\phi}$ are defined by
\begin{equation}
  \begin{aligned}
    x = r\cos\theta\sin\phi & \qquad r \ge 0 \\
    y = r\sin\theta\sin\phi & \qquad 0 \le \theta \le 2\pi \\
    z = r\cos\phi & \qquad 0 \le \phi \le \pi
  \end{aligned}
\end{equation}

The base vectors with respect to the spherical polar coordinate system are
\begin{equation}
  \vect{g}_{i}=\begin{bmatrix} 
    \cos\theta\sin\phi\vect{i}_{1}+\sin\theta\sin\phi\vect{i}_{2}+\cos\phi\vect{i}_{3} \\ 
    -r\sin\theta\sin\phi\vect{i}_{1}+r\cos\theta\sin\phi\vect{i}_{2} \\
    r\cos\theta\cos\phi\vect{i}_{1}+r\sin\theta\cos\phi\vect{i}_{2}-r\sin\phi\vect{i}_{3}
  \end{bmatrix}
\end{equation}

The covariant metric tensor is
\begin{equation}
  g_{ij}=\begin{bmatrix}
    1 & 0 & 0 \\
    0 & r^{2}\sin^{2}\phi & 0 \\
    0 & 0 & r^{2} 
  \end{bmatrix}
\end{equation}
and the contravariant metric tensor is
\begin{equation}
  g^{ij}=\begin{bmatrix}
    1 & 0 & 0 \\
    0 &  \frac{1}{r^{2}\sin^{2}\phi} & 0 \\
    0 & 0 & \frac{1}{r^{2}} 
  \end{bmatrix}
\end{equation}

The Christoffell symbols of the second kind are
\begin{align}
  \christoffelsecond{r}{\theta}{\theta}&=-r\sin^{2}\phi \\
  \christoffelsecond{r}{\phi}{\phi}&=-r \\
  \christoffelsecond{\phi}{\theta}{\theta}&=-\sin\phi\cos\phi \\
  \christoffelsecond{\theta}{r}{\theta}=\christoffelsecond{\theta}{\theta}{r}&=\frac{1}{r} \\
  \christoffelsecond{\phi}{r}{\phi}=\christoffelsecond{\phi}{\phi}{r}&=\frac{1}{r} \\
  \christoffelsecond{\theta}{\theta}{\phi}=\christoffelsecond{\theta}{\phi}{\theta}&=\cot\phi
\end{align}
with all other Christofell symbols zero.

\subsubsection{Prolate Spheroidal}

The global coordinates $\pbrac{x,y,z}$ with respect to the prolate spheroidal
coordinates $\pbrac{\lambda,\mu,\theta}$ are defined by
\begin{equation}
  \begin{aligned}
    x = a\sinh\lambda\sin\mu\cos\theta & \qquad \lambda \ge 0 \\
    y = a\sinh\lambda\sin\mu\sin\theta & \qquad 0 \le \mu \le \pi \\
    z = a\cosh\lambda\cos\mu & \qquad 0 \le \theta \le 2\pi 
  \end{aligned}
\end{equation}
where $a\ge0$ is the focus.

The base vectors with respect to the global coordinate system are
\begin{equation}
  \vect{g}_{i}=\begin{bmatrix} 
    a\cosh\lambda\sin\mu\cos\theta\vect{i}_{1}+a\cosh\lambda\sin\mu\sin\theta\vect{i}_{2}+a\sinh\lambda\cos\mu\vect{i}_{3}\\ 
    a\sinh\lambda\cos\mu\cos\theta\vect{i}_{1}+a\sinh\lambda\cos\mu\sin\theta\vect{i}_{2}-a\cosh\lambda\sin\mu\vect{i}_{3}\\
    -a\sinh\lambda\sin\mu\sin\theta\vect{i}_{1}+a\sinh\lambda\sin\mu\cos\theta\vect{i}_{2}
  \end{bmatrix}
\end{equation}

The covariant metric tensor is
\begin{equation}
  g_{ij}=\begin{bmatrix}
    a^{2}\pbrac{\sinh^{2}\lambda+\sin^{2}\mu} & 0 & 0 \\
    0 & a^{2}\pbrac{\sinh^{2}\lambda+\sin^{2}\mu} & 0 \\
    0 & 0 & a^{2}\sinh^{2}\lambda\sin^{2}\mu 
  \end{bmatrix}
\end{equation}
and the contravariant metric tensor is
\begin{equation}
  g^{ij}=\begin{bmatrix}
    \frac{1}{a^{2}\pbrac{\sinh^{2}\lambda+\sin^{2}\mu}}& 0 & 0 \\
    0 & \frac{1}{a^{2}\pbrac{\sinh^{2}\lambda+\sin^{2}\mu}} & 0 \\
    0 & 0 & \frac{1}{a^{2}\sinh^{2}\lambda\sin^{2}\mu} 
  \end{bmatrix}
\end{equation}

The Christoffell symbols of the second kind are
\begin{align}
  \christoffelsecond{\lambda}{\lambda}{\lambda}&=\frac{\sinh\lambda\cosh\lambda}{\sinh^{2}\lambda+\sin^{2}\mu} \\
  \christoffelsecond{\lambda}{\mu}{\mu}&=\frac{-\sinh\lambda\cosh\lambda}{\sinh^{2}\lambda+\sin^{2}\mu} \\
  \christoffelsecond{\lambda}{\theta}{\theta}&=\frac{-\sinh\lambda\cosh\lambda\sin^{2}\mu}{\sinh^{2}\lambda+\sin^{2}\mu} \\
  \christoffelsecond{\lambda}{\lambda}{\mu}&=\frac{\sin\mu\cos\mu}{\sinh^{2}\lambda+\sin^{2}\mu} \\
  \christoffelsecond{\mu}{\mu}{\mu}&=\frac{\sin\mu\cos\mu}{\sinh^{2}\lambda+\sin^{2}\mu} \\
  \christoffelsecond{\mu}{\lambda}{\lambda}&=\frac{-\sin\mu\cos\mu}{\sinh^{2}\lambda+\sin^{2}\mu} \\
  \christoffelsecond{\mu}{\theta}{\theta}&=\frac{-\sinh^{2}\lambda\sin\mu\cos\mu}{\sinh^{2}\lambda+\sin^{2}\mu} \\
  \christoffelsecond{\mu}{\mu}{\lambda}&=\frac{\sinh\lambda\cosh\lambda}{\sinh^{2}\lambda+\sin^{2}\mu} \\
  \christoffelsecond{\theta}{\theta}{\lambda}&=\frac{\cosh\lambda}{\sinh\lambda} \\
  \christoffelsecond{\theta}{\theta}{\mu}&=\frac{\cos\mu}{\sin\mu} \\
 \end{align}
with all other Christofell symbols zero.

\subsubsection{Oblate Spheroidal}

The global coordinates $\pbrac{x,y,z}$ with respect to the oblate spheroidal
coordinates $\pbrac{\lambda,\mu,\theta}$  are defined by
\begin{equation}
  \begin{aligned}
    x = a\cosh\lambda\cos\mu\cos\theta & \qquad \lambda \ge 0 \\
    y = a\cosh\lambda\cos\mu\sin\theta & \qquad \frac{-\pi}{2} \le \mu \le \frac{\pi}{2} \\
    z = a\sinh\lambda\sin\mu & \qquad 0 \le \theta \le 2\pi 
  \end{aligned}
\end{equation}
where $a\ge0$ is the focus.

The base vectors with respect to the global coordinate system are
\begin{equation}
  \vect{g}_{i}=\begin{bmatrix} 
    a\sinh\lambda\cos\mu\cos\theta\vect{i}_{1}+a\sinh\lambda\cos\mu\sin\theta\vect{i}_{2}+a\cosh\lambda\sin\mu\vect{i}_{3}\\
    -a\cosh\lambda\sin\mu\cos\theta\vect{i}_{1}-a\cosh\lambda\sin\mu\sin\theta\vect{i}_{2}+a\sinh\lambda\cos\mu\vect{i}_{3}\\    
    -a\cosh\lambda\cos\mu\sin\theta\vect{i}_{1}+a\cosh\lambda\cos\mu\cos\theta\vect{i}_{2}
  \end{bmatrix}
\end{equation}

The covariant metric tensor is
\begin{equation}
  g_{ij}=\begin{bmatrix}
    a^{2}\pbrac{\sinh^{2}\lambda+\sin^{2}\mu} & 0 & 0 \\
    0 & a^{2}\pbrac{\sinh^{2}\lambda+\sin^{2}\mu} & 0 \\
    0 & 0 & a^{2}\cosh^{2}\lambda\cos^{2}\mu 
  \end{bmatrix}
\end{equation}
and the contravariant metric tensor is
\begin{equation}
  g^{ij}=\begin{bmatrix}
    \frac{1}{a^{2}\pbrac{\sinh^{2}\lambda+\sin^{2}\mu}}& 0 & 0 \\
    0 & \frac{1}{a^{2}\pbrac{\sinh^{2}\lambda+\sin^{2}\mu}} & 0 \\
    0 & 0 & \frac{1}{a^{2}\cosh^{2}\lambda\cos^{2}\mu}
  \end{bmatrix}
\end{equation}

The Christoffell symbols of the second kind are
\begin{align}
  \christoffelsecond{\lambda}{\lambda}{\lambda}&=\frac{\sinh\lambda\cosh\lambda}{\sinh^{2}\lambda+\sin^{2}\mu} \\
  \christoffelsecond{\lambda}{\mu}{\mu}&=\frac{-\sinh\lambda\cosh\lambda}{\sinh^{2}\lambda+\sin^{2}\mu} \\
  \christoffelsecond{\lambda}{\theta}{\theta}&=\frac{-\sinh\lambda\cosh\lambda\cos^{2}\mu}{\sinh^{2}\lambda+\sin^{2}\mu} \\
  \christoffelsecond{\lambda}{\lambda}{\mu}&=\frac{\sin\mu\cos\mu}{\sinh^{2}\lambda+\sin^{2}\mu} \\
  \christoffelsecond{\mu}{\mu}{\mu}&=\frac{\sin\mu\cos\mu}{\sinh^{2}\lambda+\sin^{2}\mu} \\
  \christoffelsecond{\mu}{\lambda}{\lambda}&=\frac{-\sin\mu\cos\mu}{\sinh^{2}\lambda+\sin^{2}\mu} \\
  \christoffelsecond{\mu}{\theta}{\theta}&=\frac{\cosh^{2}\lambda\sin\mu\cos\mu}{\sinh^{2}\lambda+\sin^{2}\mu} \\
  \christoffelsecond{\mu}{\mu}{\lambda}&=\frac{\sinh\lambda\cosh\lambda}{\sinh^{2}\lambda+\sin^{2}\mu} \\
  \christoffelsecond{\theta}{\theta}{\lambda}&=\frac{\sinh\lambda}{\cosh\lambda} \\
  \christoffelsecond{\theta}{\theta}{\mu}&=\frac{-\sin\mu}{\cos\mu} \\
\end{align}
with all other Christofell symbols zero.

\subsubsection{Cylindrical parabolic}

The global coordinates $\pbrac{x,y,z}$ with respect to the cylindrical parabolic
coordinates $\pbrac{\xi,\eta,z}$  are defined by
\begin{equation}
  \begin{aligned}
    x = \xi\eta & \qquad -\infty < \xi < \infty \\
    y = \frac{1}{2}\pbrac{\xi^{2}-\eta^{2}} & \qquad \eta \ge 0 \\
    z =  z & \qquad -\infty < z < \infty
  \end{aligned}
\end{equation}

The base vectors with respect to the global coordinate system are
\begin{equation}
  \vect{g}_{i}=\begin{bmatrix} 
    \eta\vect{i}_{1}+\xi\vect{i}_{2}\\
    \xi\vect{i}_{1}-\eta\vect{i}_{2}\\    
    \vect{i}_{3}
  \end{bmatrix}
\end{equation}

The covariant metric tensor is
\begin{equation}
  g_{ij}=\begin{bmatrix}
    \xi^{2}+\eta^{2} & 0 & 0 \\
    0 & \xi^{2}+\eta^{2} & 0 \\
    0 & 0 & 1
  \end{bmatrix}
\end{equation}
and the contravariant metric tensor is
\begin{equation}
  g^{ij}=\begin{bmatrix}
    \frac{1}{\xi^{2}+\eta^{2}}& 0 & 0 \\
    0 & \frac{1}{\xi^{2}+\eta^{2}} & 0 \\
    0 & 0 & 1
  \end{bmatrix}
\end{equation}

The Christoffell symbols of the second kind are
\begin{align}
  \christoffelsecond{\xi}{\xi}{\xi}&=\frac{\xi}{\xi^{2}+\eta^{2}} \\
  \christoffelsecond{\eta}{\eta}{\eta}&=\frac{\eta}{\xi^{2}+\eta^{2}} \\
  \christoffelsecond{\eta}{\xi}{\xi}&=\frac{-\eta}{\xi^{2}+\eta^{2}} \\
  \christoffelsecond{\xi}{\eta}{\eta}&=\frac{-\xi}{\xi^{2}+\eta^{2}} \\
  \christoffelsecond{\xi}{\xi}{\eta}=\christoffelsecond{\xi}{\eta}{\xi}&=\frac{\eta}{\xi^{2}+\eta^{2}} \\
  \christoffelsecond{\eta}{\xi}{\eta}=\christoffelsecond{\eta}{\eta}{\xi}&=\frac{\xi}{\xi^{2}+\eta^{2}} \\
\end{align}
with all other Christofell symbols zero.

\subsubsection{Parabolic polar}

The global coordinates $\pbrac{x,y,z}$ with respect to the cylindrical parabolic
coordinates $\pbrac{\xi,\eta,\theta}$  are defined by
\begin{equation}
  \begin{aligned}
    x = \xi\eta\cos\theta & \qquad \xi \ge 0 \\
    y = \xi\eta\sin\theta & \qquad \eta \ge 0 \\
    z = \frac{1}{2}\pbrac{\xi^{2}-\eta^{2}} & \qquad 0 \le \theta < 2\pi
  \end{aligned}
\end{equation}

The base vectors with respect to the global coordinate system are
\begin{equation}
  \vect{g}_{i}=\begin{bmatrix} 
    \eta\cos\theta\vect{i}_{1}+\eta\sin\theta\vect{i}_{3}+\xi\vect{i}_{3}\\
    \xi\cos\theta\vect{i}_{1}+\xi\sin\theta\vect{i}_{3}-\eta\vect{i}_{3}\\ 
    -\xi\eta\sin\theta\vect{i}_{1}+\xi\eta\cos\theta\vect{i}_{2}
  \end{bmatrix}
\end{equation}

The covariant metric tensor is
\begin{equation}
  g_{ij}=\begin{bmatrix}
    \xi^{2}+\eta^{2} & 0 & 0 \\
    0 & \xi^{2}+\eta^{2} & 0 \\
    0 & 0 & \xi\eta
  \end{bmatrix}
\end{equation}
and the contravariant metric tensor is
\begin{equation}
  g^{ij}=\begin{bmatrix}
    \frac{1}{\xi^{2}+\eta^{2}}& 0 & 0 \\
    0 & \frac{1}{\xi^{2}+\eta^{2}} & 0 \\
    0 & 0 & \frac{1}{\xi\eta}
  \end{bmatrix}
\end{equation}

The Christoffell symbols of the second kind are
\begin{align}
  \christoffelsecond{\xi}{\xi}{\xi}&=\frac{\xi}{\xi^{2}+\eta^{2}} \\
  \christoffelsecond{\eta}{\eta}{\eta}&=\frac{\eta}{\xi^{2}+\eta^{2}} \\
  \christoffelsecond{\xi}{\eta}{\eta}&=\frac{-\xi}{\xi^{2}+\eta^{2}} \\
  \christoffelsecond{\eta}{\xi}{\xi}&=\frac{-\eta}{\xi^{2}+\eta^{2}} \\
  \christoffelsecond{\eta}{\theta}{\theta}&=\frac{-\xi^{2}\eta}{\xi^{2}+\eta^{2}} \\
  \christoffelsecond{\xi}{\theta}{\theta}&=\frac{-\xi\eta^{2}}{\xi^{2}+\eta^{2}} \\
  \christoffelsecond{\xi}{\xi}{\eta}=\christoffelsecond{\xi}{\eta}{\xi}&=\frac{\eta}{\xi^{2}+\eta^{2}} \\
  \christoffelsecond{\eta}{\xi}{\eta}=\christoffelsecond{\eta}{\eta}{\xi}&=\frac{\xi}{\xi^{2}+\eta^{2}} \\
  \christoffelsecond{\theta}{\xi}{\theta}=\christoffelsecond{\theta}{\theta}{\xi}&=\frac{1}{\xi} \\
  \christoffelsecond{\theta}{\eta}{\theta}=\christoffelsecond{\theta}{\theta}{\eta}&=\frac{1}{\eta} \\
\end{align}
with all other Christofell symbols zero.

\section{Equation set types}

\subsection{Static Equations}

The general form for static equations is

\subsection{Dynamic Equations}

The general form for dynamic equations is
\begin{equation}
  \matr{M}\fnof{\ddot{\vect{u}}}{t}+\matr{C}\fnof{\dot{\vect{u}}}{t}+\matr{K}\fnof{\vect{u}}{t}+
  \fnof{\vect{g}}{\fnof{\vect{u}}{t}}+\fnof{\vect{f}}{t}=\vect{0}
  \label{eqn:generaldynamicnonlinear}
\end{equation}
where $\fnof{\vect{u}}{t}$ is the unknown ``displacement vector'', $\matr{M}$
is the mass matrix, $\matr{C}$ is the damping matrix, $\matr{K}$ is the
stiffness matrix, $\fnof{\vect{g}}{\fnof{\vect{u}}{t}}$ a non-linear vector
function and $\fnof{\vect{f}}{t}$ the forcing vector.

From \cite{zienkiewicz:2006_1} we now expand the unknown vector $\fnof{\vect{u}}{t}$ in terms of a polynomial of degree
$p$. With the known values of $\vect{u}_{n}$, $\dot{\vect{u}}_{n}$,
$\ddot{\vect{u}}_{n}$ up to $\symover{p-1}{\vect{u}}_{n}$ at the beginning of
the time step $\Delta t$ we can write the polynomial expansion as
\begin{equation}
  \fnof{\vect{u}}{t_{n}+\tau}\approx\fnof{\tilde{\vect{u}}}{t_{n}+\tau}=\vect{u}_{n}+\tau\dot{\vect{u}}_{n}+
  \frac{1}{2!}\tau^{2}\ddot{\vect{u}}_{n}+\cdots+\dfrac{1}{\factorial{p-1}}\tau^{p-1}\symover{p-1}{\vect{u}}_{n}+
  \dfrac{1}{p!}\tau^{p}\vect{\alpha}^{p}_{n}
  \label{eqn:timepolyexpansion}
\end{equation}
where the only unknown is the the vector $\vect{\alpha}^{p}_{n}$,
\begin{equation}
  \vect{\alpha}^{p}_{n}\approx\symover{p}{\vect{u}}\equiv\dnby{p}{\vect{u}}{t}
\end{equation}

A recurrance relationship can be established by substituting
\eqnref{eqn:timepolyexpansion} into \eqnref{eqn:generaldynamicnonlinear} and
taking a weighted residual approach \ie
\begin{multline}
  \dintl{0}{\Delta
    t}\fnof{W}{\tau}\left[\matr{M}\pbrac{\ddot{\vect{u}}_{n}+\tau\dddot{\vect{u}}_{n}+\cdots+
    \dfrac{1}{\factorial{p-2}}\tau^{p-2}\vect{\alpha}^{p}_{n}} \right.\\
  +\matr{C}\pbrac{\dot{\vect{u}}_{n}+\tau\ddot{\vect{u}}_{n}+\cdots+
    \dfrac{1}{\factorial{p-1}}\tau^{p-1}\vect{\alpha}^{p}_{n}} \\
  +\matr{K}\pbrac{\vect{u}_{n}+\tau\dot{\vect{u}}_{n}+\cdots+
    \dfrac{1}{p!}\tau^{p}\vect{\alpha}^{p}_{n}} \\
  +\left.\fnof{\vect{g}}{\vect{u}_{n}+\tau\dot{\vect{u}}_{n}+\cdots+
    \dfrac{1}{p!}\tau^{p}\vect{\alpha}^{p}_{n}}+\fnof{\vect{f}}{t_{n}+\tau}\right] d\tau = \vect{0}
\end{multline}
where $\fnof{W}{\tau}$ is some weight function, $\tau=t-t_{n}$ and $\Delta
t=t_{n+1}-t_{n}$. Dividing by $\gint{0}{\Delta t}{\fnof{W}{\tau}}{\tau}$ we obtain
\begin{multline}
  \dfrac{\gint{0}{\Delta t}{\fnof{W}{\tau}\matr{M}\pbrac{\ddot{\vect{u}}_{n}+\tau\dddot{\vect{u}}_{n}+\cdots+
        \dfrac{1}{\factorial{p-2}}\tau^{p-2}\vect{\alpha}^{p}_{n}}}{\tau}}{\gint{0}{\Delta
      t}{\fnof{W}{\tau}}{\tau}} \\
  + \dfrac{\gint{0}{\Delta t}{\fnof{W}{\tau}\matr{C}\pbrac{\dot{\vect{u}}_{n}+\tau\ddot{\vect{u}}_{n}+\cdots+
        \dfrac{1}{\factorial{p-1}}\tau^{p-1}\vect{\alpha}^{p}_{n}}}{\tau}}{\gint{0}{\Delta
      t}{\fnof{W}{\tau}}{\tau}} \\
  + \dfrac{\gint{0}{\Delta t}{\fnof{W}{\tau}\matr{K}\pbrac{\vect{u}_{n}+\tau\dot{\vect{u}}_{n}+\cdots+
        \dfrac{1}{p!}\tau^{p}\vect{\alpha}^{p}_{n}}}{\tau}}{\gint{0}{\Delta
      t}{\fnof{W}{\tau}}{\tau}} \\
  + \dfrac{\gint{0}{\Delta t}{\fnof{W}{\tau}\fnof{\vect{g}}{\vect{u}_{n}+\tau\dot{\vect{u}}_{n}+\cdots+
        \dfrac{1}{p!}\tau^{p}\vect{\alpha}^{p}_{n}}}{\tau}}{\gint{0}{\Delta
      t}{\fnof{W}{\tau}}{\tau}}  
  + \dfrac{\gint{0}{\Delta t}{\fnof{W}{\tau}\fnof{\vect{f}}{t_{n}+
        \tau}}{\tau}}{\gint{0}{\Delta t}{\fnof{W}{\tau}}{\tau}}=\vect{0}
\end{multline}

Now if 
\begin{equation}
  \theta_{k}=\dfrac{\gint{0}{\Delta t}{\fnof{W}{\tau}\tau^{k}}{\tau}}{{\Delta
      t}^{k}\gint{0}{\Delta t}{\fnof{W}{\tau}}{\tau}} \text{  for  } k=0,1,\ldots,p
\end{equation}
and
\begin{equation}
  \bar{\vect{f}}=\dfrac{\gint{0}{\Delta
      t}{\fnof{W}{\tau}\fnof{\vect{f}}{t_{n}+\tau}}{\tau}}{
    \gint{0}{\Delta t}{\fnof{W}{\tau}}{\tau}}
  \label{eqn:meanweightedloadvector}
\end{equation}
we can write
\begin{multline}
  \matr{M}\pbrac{\ddot{\bar{\vect{u}}}_{n+1}+\dfrac{\theta_{p-2}{\Delta
        t}^{p-2}}{\factorial{p-2}}\vect{\alpha}^{p}_{n}}+
  \matr{C}\pbrac{\dot{\bar{\vect{u}}}_{n+1}+\dfrac{\theta_{p-1}{\Delta
        t}^{p-1}}{\factorial{p-1}}\vect{\alpha}^{p}_{n}}+
  \matr{K}\pbrac{\bar{\vect{u}}_{n+1}+\dfrac{\theta_{p}{\Delta
        t}^{p}}{p!}\vect{\alpha}^{p}_{n}}+ \\
  + \dfrac{\gint{0}{\Delta t}{\fnof{W}{\tau}\fnof{\vect{g}}{\vect{u}_{n}+\tau\dot{\vect{u}}_{n}+\cdots+
        \dfrac{1}{p!}\tau^{p}\vect{\alpha}^{p}_{n}}}{\tau}}{\gint{0}{\Delta
      t}{\fnof{W}{\tau}}{\tau}}+\bar{\vect{f}}=\vect{0}
  \label{eqn:dynamic1}
\end{multline}
where
\begin{equation}
  \begin{split}
    \bar{\vect{u}}_{n+1} &= \gsum{q=0}{p-1}{\dfrac{\theta_{q}{\Delta
            t}^{q}}{q!}\symover{q}{\vect{u}}_{n}} \\
    \dot{\bar{\vect{u}}}_{n+1} &= \gsum{q=1}{p-1}{\dfrac{\theta_{q-1}{\Delta
            t}^{q-1}}{\factorial{q-1}}\symover{q}{\vect{u}}_{n}} \\
    \ddot{\bar{\vect{u}}}_{n+1} &= \gsum{q=2}{p-1}{\dfrac{\theta_{q-2}{\Delta
            t}^{q-2}}{\factorial{q-2}}\symover{q}{\vect{u}}_{n}} 
  \end{split}
\end{equation}

We note that as $\fnof{\vect{g}}{\fnof{\vect{u}}{t}}$ is nonlinear we need to
evaluate an integral of the form
\begin{equation}
  \gint{0}{\Delta t}{\fnof{W}{\tau}\fnof{\vect{g}}{\fnof{\vect{u}}{t_{n}+\tau}}}{\tau}
\end{equation}

To do this we form Taylor's series expansions for
$\fnof{\vect{g}}{\fnof{\vect{u}}{t}}$ about the point $\fnof{\vect{u}}{t_{n}+\tau}$ \ie
\begin{equation}
  \fnof{\vect{g}}{\fnof{\vect{u}}{t_{n}}}=\fnof{\vect{g}}{\fnof{\vect{u}}{t_{n}+\tau}}-
  \tau\delby{\fnof{\vect{g}}{\fnof{\vect{u}}{t}}}{\vect{u}}\evalat{\delby{\fnof{\vect{u}}{t}}{t}}{t_{n}+\tau}
  + \orderof{\tau^{2}}
  \label{eqn:firstTaylorexpansion}
\end{equation}
and
\begin{equation}
  \fnof{\vect{g}}{\fnof{\vect{u}}{t_{n+1}}}=\fnof{\vect{g}}{\fnof{\vect{u}}{t_{n}+\tau}}+
  \pbrac{t_{n+1}-t_{n}-\tau}\delby{\fnof{\vect{g}}{\fnof{\vect{u}}{t}}}{\vect{u}}
  \evalat{\delby{\fnof{\vect{u}}{t}}{t}}{t_{n}+\tau}+ \orderof{\tau^{2}}
  \label{eqn:secondTaylorexpansion}
\end{equation}

Now if we add $\dfrac{1}{\tau}$ times \eqnref{eqn:firstTaylorexpansion} and
$\dfrac{1}{t_{n+1}-t_{n}-\tau}=\dfrac{1}{\Delta t-\tau}$ times
\eqnref{eqn:secondTaylorexpansion} we obtain
\begin{equation}
  \dfrac{\fnof{\vect{g}}{\fnof{\vect{u}}{t_{n}}}}{\tau}+\dfrac{\fnof{\vect{g}}{\fnof{\vect{u}}{t_{n+1}}}}{\Delta
    t-\tau}=\pbrac{\dfrac{\Delta t}{\tau\pbrac{\Delta t-\tau}}}\fnof{\vect{g}}{\fnof{\vect{u}}{t_{n}+\tau}}+
  \pbrac{\dfrac{\Delta t}{\tau\pbrac{\Delta t-\tau}}}\orderof{\tau^{2}}
\end{equation}

Multiplying through by $\dfrac{\tau\pbrac{\Delta t-\tau}}{\Delta t}$ gives
\begin{equation}
  \dfrac{\Delta t-\tau}{\Delta t}\fnof{\vect{g}}{\fnof{\vect{u}}{t_{n}}}+
  \dfrac{\tau}{\Delta t}\fnof{\vect{g}}{\fnof{\vect{u}}{t_{n+1}}}=
  \fnof{\vect{g}}{\fnof{\vect{u}}{t_{n}+\tau}}+\orderof{\tau^{2}}
\end{equation}

Therefore
\begin{equation}
  \dfrac{\gint{0}{\Delta t}{\fnof{W}{\tau}\fnof{\vect{g}}{\fnof{\vect{u}}{t_{n}+\tau}}}{\tau}}
  {\gint{0}{\Delta t}{\fnof{W}{\tau}}{\tau}}=\dfrac{\gint{0}{\Delta t}{\fnof{W}{\tau}
      \pbrac{\dfrac{\Delta t-\tau}{\Delta t}\fnof{\vect{g}}{\fnof{\vect{u}}{t_{n}}}+
        \dfrac{\tau}{\Delta t}\fnof{\vect{g}}{\fnof{\vect{u}}{t_{n+1}}}+\orderof{\tau^{2}}}}{\tau}}
  {\gint{0}{\Delta t}{\fnof{W}{\tau}}{\tau}}
\end{equation}

Now if we recall that
\begin{equation}
\theta_{1}=\dfrac{\gint{0}{\Delta t}{\fnof{W}{\tau}\tau}{\tau}}{\Delta t\gint{0}{\Delta t}{\fnof{W}{\tau}}{\tau}}
\end{equation}
we can write
\begin{equation}
  \dfrac{\gint{0}{\Delta t}{\fnof{W}{\tau}\fnof{\vect{g}}{\fnof{\vect{u}}{t_{n+1}}}}{\tau}}
  {\gint{0}{\Delta t}{\fnof{W}{\tau}}{\tau}}=\pbrac{1-\theta_{1}}\fnof{\vect{g}}{\fnof{\vect{u}}{t_{n}}}+
  \theta_{1}\fnof{\vect{g}}{\fnof{\vect{u}}{t_{n+1}}}+\text{Error}
\end{equation}
where
\begin{equation}
  \text{Error}=\dfrac{\gint{0}{\Delta t}{\fnof{W}{\tau}\orderof{\tau^{2}}}{\tau}}{
    \gint{0}{\Delta t}{\fnof{W}{\tau}}{\tau}}
\end{equation}

\Eqnref{eqn:dynamic1} now becomes
\begin{multline}
  \matr{M}\pbrac{\ddot{\bar{\vect{u}}}_{n+1}+\dfrac{\theta_{p-2}{\Delta
        t}^{p-2}}{\factorial{p-2}}\vect{\alpha}^{p}_{n}}+
  \matr{C}\pbrac{\dot{\bar{\vect{u}}}_{n+1}+\dfrac{\theta_{p-1}{\Delta
        t}^{p-1}}{\factorial{p-1}}\vect{\alpha}^{p}_{n}}\\
  +\matr{K}\pbrac{\bar{\vect{u}}_{n+1}+\dfrac{\theta_{p}{\Delta
        t}^{p}}{p!}\vect{\alpha}^{p}_{n}}+ 
  \pbrac{1-\theta_{1}}\fnof{\vect{g}}{\vect{u}_{n}}+\theta_{1}\fnof{\vect{g}}{\vect{u}_{n+1}}+\bar{\vect{f}}+
  \text{Error}=\vect{0}
  \label{eqn:dynamic2}
\end{multline}
as $\fnof{\vect{u}}{t_{n}}=\vect{u}_{n}$ and
$\fnof{\vect{u}}{t_{n+1}}=\vect{u}_{n+1}=\hat{\vect{u}}_{n+1}+
\dfrac{{\Delta t}^{p}}{p!}\vect{\alpha}^{p}_{n}$ where $\hat{\vect{u}}_{n+1}$
is the \emph{predicted displacement} at the new time step and is given by
\begin{equation}
  \hat{\vect{u}}_{n+1}=\gsum{q=0}{p-1}{\dfrac{{\Delta
        t}^{q}}{q!}\symover{q}{\vect{u}}_{n}}
\end{equation}

Rearranging gives
\begin{multline}
  \fnof{\vect{\psi}}{\vect{\alpha}^{p}_{n}}=\pbrac{\dfrac{\theta_{p-2}{\Delta
        t}^{p-2}}{\factorial{p-2}}\matr{M}+\dfrac{\theta_{p-1}{\Delta
        t}^{p-1}}{\factorial{p-1}}\matr{C}+\dfrac{\theta_{p}{\Delta
        t}^{p}}{p!}\matr{K}}\vect{\alpha}^{p}_{n}+\theta_{1}\fnof{\vect{g}}{\hat{\vect{u}}_{n+1}+ 
    \dfrac{{\Delta t}^{p}}{p!}\vect{\alpha}^{p}_{n}} \\
  +\pbrac{1-\theta_{1}}\fnof{\vect{g}}{\vect{u}_{n}}+
  \pbrac{\matr{M}\ddot{\bar{\vect{u}}}_{n+1}+\matr{C}\dot{\bar{\vect{u}}}_{n+1}+\matr{K}\bar{\vect{u}}_{n+1}+
    \bar{\vect{f}}}= \vect{0}
  \label{eqn:dynamic}
\end{multline}
or 
\begin{equation}
\fnof{\vect{\psi}}{\vect{\alpha}^{p}_{n}}=\matr{A}\vect{\alpha}^{p}_{n}+
\theta_{1}\fnof{\vect{g}}{\hat{\vect{u}}_{n+1}+ \dfrac{{\Delta
      t}^{p}}{p!}\vect{\alpha}^{p}_{n}}+\pbrac{1-\theta_{1}}\fnof{\vect{g}}{\vect{u}_{n}}+\vect{b}= \vect{0}
\end{equation}
where $\matr{A}$ is the \emph{Amplification matrix} given by
\begin{equation}
  \matr{A}=\dfrac{\theta_{p-2}{\Delta t}^{p-2}}{\factorial{p-2}}\matr{M}+
  \dfrac{\theta_{p-1}{\Delta t}^{p-1}}{\factorial{p-1}}\matr{C}+
  \dfrac{\theta_{p}{\Delta t}^{p}}{p!}\matr{K}
\end{equation}
and $\vect{b}$ is the right hand side vector given by
\begin{equation}
  \vect{b}=\matr{M}\ddot{\bar{\vect{u}}}_{n+1}+\matr{C}\dot{\bar{\vect{u}}}_{n+1}+
  \matr{K}\bar{\vect{u}}_{n+1}+\bar{\vect{f}}
\end{equation}

If $\fnof{\vect{g}}{\vect{u}}\equiv\vect{0}$ then \eqnref{eqn:dynamic} is linear in
$\vect{\alpha}^{p}_{n}$ and $\vect{\alpha}^{p}_{n}$ can be found by solving
the linear equation
\begin{equation}
  \vect{\alpha}^{p}_{n} =-\inverse{\pbrac{\dfrac{\theta_{p-2}{\Delta t}^{p-2}}{\factorial{p-2}}\matr{M}+
      \dfrac{\theta_{p-1}{\Delta t}^{p-1}}{\factorial{p-1}}\matr{C}+
      \dfrac{\theta_{p}{\Delta
          t}^{p}}{p!}\matr{K}}}\pbrac{\matr{M}\ddot{\bar{\vect{u}}}_{n+1}+
    \matr{C}\dot{\bar{\vect{u}}}_{n+1}+\matr{K}\bar{\vect{u}}_{n+1}+\bar{\vect{f}}}
\end{equation}
or 
\begin{equation}
  \vect{\alpha}^{p}_{n} =-\inverse{\matr{A}}\vect{b}
\end{equation}

If $\fnof{\vect{g}}{\vect{u}}$ is not $\equiv\vect{0}$ then
\eqnref{eqn:dynamic} is nonlinear in $\vect{\alpha}^{p}_{n}$. To solve this
equation we use Newton's method \ie
\begin{equation}
  \begin{split}
    \text{1.  } & \fnof{\matr{J}}{\vect{\alpha}^{p}_{n(i)}}.\delta
    \vect{\alpha}^{p}_{n(i)} = 
    -\fnof{\vect{\psi}}{\vect{\alpha}^{p}_{n(i)}} \\
    \text{2.  } & \vect{\alpha}^{p}_{n(i+1)}=\vect{\alpha}^{p}_{n(i)}+\delta
    \vect{\alpha}^{p}_{n(i)}
  \end{split}
\end{equation}
where $\fnof{\matr{J}}{\vect{\alpha}^{p}_{n}}$ is the Jacobian and is given by
\begin{equation}
  \fnof{\matr{J}}{\vect{\alpha}^{p}_{n}}=\dfrac{\theta_{p-2}{\Delta t}^{p-2}}{\factorial{p-2}}\matr{M}+
  \dfrac{\theta_{p-1}{\Delta
      t}^{p-1}}{\factorial{p-1}}\matr{C}+\dfrac{\theta_{p}{\Delta t}^{p}}{p!}\matr{K}+
  \dfrac{\theta_{1}{\Delta t}^{p}}{p!}
  \delby{\fnof{\vect{g}}{\hat{\vect{u}}_{n+1}+\dfrac{{\Delta
          t}^{p}}{p!}
      \vect{\alpha}^{p}_{n}}}{\vect{\alpha}^{p}_{n}}
\end{equation}
or
\begin{equation}
  \fnof{\matr{J}}{\vect{\alpha}^{p}_{n}}=\matr{A}+\dfrac{\theta_{1}{\Delta
      t}^{p}}{p!}
  \delby{\fnof{\vect{g}}{\hat{\vect{u}}_{n+1}+\dfrac{{\Delta t}^{p}}{p!}\vect{\alpha}^{p}_{n}}}{\vect{\alpha}^{p}_{n}}
\end{equation}

Once $\vect{\alpha}^{p}_{n}$ has been obtained the values at the next time step can be obtained from
\begin{equation}
  \begin{split}
    \vect{u}_{n+1} &= \vect{u}_{n}+\Delta t
    \dot{\vect{u}}_{n}+\cdots+\dfrac{{\Delta
        t}^{p}}{p!}\vect{\alpha}^{p}_{n}=\hat{\vect{u}}_{n+1}+
    \dfrac{{\Delta t}^{p}}{p!}\vect{\alpha}^{p}_{n}\\
    \dot{\vect{u}}_{n+1} &= \dot{\vect{u}}_{n}+\Delta t
    \ddot{\vect{u}}_{n}+\cdots+\dfrac{{\Delta
        t}^{p-1}}{\factorial{p-1}}\vect{\alpha}^{p}_{n}=\dot{\hat{\vect{u}}}_{n+1}+\dfrac{{\Delta
        t}^{p-1}}{\factorial{p-1}}\vect{\alpha}^{p}_{n} \\
    &\vdots \\
    \symover{p-1}{\vect{u}}_{n+1} &= \symover{p-1}{\vect{u}}_{n}+\Delta t\vect{\alpha}^{p}_{n}
  \end{split}
\end{equation}

For algorithms in which the degree of the polynomial, $p$, is higher than the
order we require the algorithm to be initialised so that the initial velocity
or acceleration can be computed. The initial velocity or acceleration values
can be obtained by substituting the initial displacement or initial
displacement and velocity values into \eqnref{eqn:generaldynamicnonlinear},
rearranging and solving. For example consider an the case of a second degree
polynomial and a first order system. Substituing the initial displacement
$\vect{u}_{0}$ into \eqnref{eqn:generaldynamicnonlinear} gives
\begin{equation}
  \matr{C}\dot{\vect{u}}_{0}+\matr{K}\vect{u}_{0}+\fnof{\vect{g}}{\vect{u}_{0}}+\bar{\vect{f}}_{0}=\vect{0}
\end{equation}
and therefore an approximation to the initial velocity can be found from
\begin{equation}
  \dot{\vect{u}}_{0}=-\inverse{\matr{C}}\pbrac{\matr{K}\vect{u}_{0}+\fnof{\vect{g}}{\vect{u}_{0}}+\bar{\vect{f}}_{0}}
\end{equation}

Similarily for a third degree polynomial and a second order system the initial
acceleration can be found from
\begin{equation}
  \ddot{\vect{u}}_{0}=-\inverse{\matr{M}}\pbrac{\matr{C}\dot{\vect{u}}_{0}+\matr{K}\vect{u}_{0}+
    \fnof{\vect{g}}{\vect{u}_{0}}+\bar{\vect{f}}_{0}}
\end{equation}

To evaluate the mean weighted load vector, $\bar{\vect{f}}$, we need to
evaluate the integral in \eqnref{eqn:meanweightedloadvector}. In some cases,
however, we can make the assumption that the load vector varies linearly
during the time step. In these cases the mean weighted load vector can be
computed from
\begin{equation}
  \bar{\vect{f}}=\theta_{1}\vect{f}_{n+1}+\pbrac{1-\theta_{1}}\vect{f}_{n}
\end{equation}

\subsubsection{Special SN11 case, p=1}

For this special case, the mean predicited values are given by
\begin{equation}
   \bar{\vect{u}}_{n+1} = \vect{u}_{n}
\end{equation}

The predicted displacement values are given by
\begin{equation}
   \hat{\vect{u}}_{n+1} = \vect{u}_{n}
\end{equation}

The amplification matrix is given by
\begin{equation}
  \matr{A}=\matr{C}+\theta_{1}\Delta t \matr{K}
\end{equation}

The right hand side vector is given by
\begin{equation}
  \vect{b}=\matr{K}\bar{\vect{u}}_{n+1}+\bar{\vect{f}}
\end{equation}

The nonlinear function is given by
\begin{equation}
  \fnof{\vect{\psi}}{\vect{\alpha}^{1}_{n}}=\matr{A}\vect{\alpha}^{1}_{n}+\theta_{1}\fnof{\vect{g}}{\hat{\vect{u}}_{n+1}+ 
    \Delta t\vect{\alpha}^{1}_{n}}+\pbrac{1-\theta_{1}}\fnof{\vect{g}}{\vect{u}_{n}}+\vect{b}=\vect{0}
\end{equation}

The Jacobian matrix is given by
\begin{equation}
  \fnof{\matr{J}}{\vect{\alpha}^{1}_{n}}=\matr{A}+\theta_{1}\Delta t
  \delby{\fnof{\vect{g}}{\hat{\vect{u}}_{n+1}+\Delta t\vect{\alpha}^{1}_{n}}}{\vect{\alpha}^{1}_{n}}
\end{equation}

And the time step update is given by
\begin{equation}
    \vect{u}_{n+1} = \vect{u}_{n}+\Delta t\vect{\alpha}^{1}_{n}
\end{equation}

\subsubsection{Special SN21 case, p=2}

For this special case, the mean predicited values are given by
\begin{equation}
  \begin{split}
    \bar{\vect{u}}_{n+1} &= \vect{u}_{n}+\theta_{1}\Delta t\dot{\vect{u}}_{n}\\
    \dot{\bar{\vect{u}}}_{n+1} &= \dot{\vect{u}}_{n}
  \end{split}
\end{equation}
where
\begin{equation}
  \dot{\vect{u}}_{0}=-\inverse{\matr{C}}\pbrac{\matr{K}\vect{u}_{0}+\fnof{\vect{g}}{\vect{u}_{0}}+\bar{\vect{f}}_{0}}
\end{equation}

The predicted displacement values are given by
\begin{equation}
   \hat{\vect{u}}_{n+1} = \vect{u}_{n}+\Delta t\dot{\vect{u}}_{n}
\end{equation}

The amplification matrix is given by
\begin{equation}
  \matr{A}=\theta_{1}\Delta t\matr{C}+\dfrac{\theta_{2}{\Delta t}^{2}}{2}\matr{K}
\end{equation}

The right hand side vector is given by
\begin{equation}
  \vect{b}=\matr{C}\dot{\bar{\vect{u}}}_{n+1}+\matr{K}\bar{\vect{u}}_{n+1}+\bar{\vect{f}}
\end{equation}

The nonlinear function is given by
\begin{equation}
  \fnof{\vect{\psi}}{\vect{\alpha}^{2}_{n}}=\matr{A}\vect{\alpha}^{2}_{n}+\theta_{1}\fnof{\vect{g}}{\hat{\vect{u}}_{n+1}+
    \dfrac{{\Delta t}^{2}}{2}\vect{\alpha}^{2}_{n}}+\pbrac{1-\theta_{1}}\fnof{\vect{g}}{\vect{u}_{n}}+\vect{b}=\vect{0}
\end{equation}

The Jacobian matrix is given by
\begin{equation}
  \fnof{\matr{J}}{\vect{\alpha}^{2}_{n}}=\matr{A}+\dfrac{\theta_{1}{\Delta t}^{2}}{2}
  \delby{\fnof{\vect{g}}{\hat{\vect{u}}_{n+1}+\dfrac{{\Delta t}^{2}}{2}\vect{\alpha}^{2}_{n}}}{\vect{\alpha}^{2}_{n}}
\end{equation}

And the time step update is given by
\begin{equation}
  \begin{split}
    \vect{u}_{n+1} &= \vect{u}_{n}+\Delta t\dot{\vect{u}}_{n} +\dfrac{{\Delta t}^{2}}{2}\vect{\alpha}^{2}_{n} \\
    \dot{\vect{u}}_{n+1} &= \dot{\vect{u}}_{n}+\Delta t\vect{\alpha}^{2}_{n}
  \end{split}
\end{equation}

\subsubsection{Special SN22 case, p=2}

For this special case, the mean predicited values are given by
\begin{equation}
  \begin{split}
    \bar{\vect{u}}_{n+1} &= \vect{u}_{n}+\theta_{1}\Delta t\dot{\vect{u}}_{n}\\
    \dot{\bar{\vect{u}}}_{n+1} &= \dot{\vect{u}}_{n}
  \end{split}
\end{equation}

The predicted displacement values are given by
\begin{equation}
   \hat{\vect{u}}_{n+1} = \vect{u}_{n}+\Delta t\dot{\vect{u}}_{n}
\end{equation}

The amplification matrix is given by
\begin{equation}
  \matr{A}=\matr{M}+\theta_{1}\Delta t\matr{C}+\dfrac{\theta_{2}{\Delta t}^{2}}{2}\matr{K}
\end{equation}

The right hand side vector is given by
\begin{equation}
  \vect{b}=\matr{C}\dot{\bar{\vect{u}}}_{n+1}+\matr{K}\bar{\vect{u}}_{n+1}+\bar{\vect{f}}
\end{equation}

The nonlinear function is given by
\begin{equation}
  \fnof{\vect{\psi}}{\vect{\alpha}^{2}_{n}}=\matr{A}\vect{\alpha}^{2}_{n}+\theta_{1}\fnof{\vect{g}}{\hat{\vect{u}}_{n+1}+ 
    \dfrac{{\Delta t}^{2}}{2}\vect{\alpha}^{2}_{n}}+\pbrac{1-\theta_{1}}\fnof{\vect{g}}{\vect{u}_{n}}+\vect{b}=\vect{0}
\end{equation}

The Jacobian matrix is given by
\begin{equation}
  \fnof{\matr{J}}{\vect{\alpha}^{2}_{n}}=\matr{A}+\dfrac{\theta_{1}{\Delta t}^{2}}{2}
  \delby{\fnof{\vect{g}}{{\hat{\vect{u}}_{n+1}+\dfrac{{\Delta t}^{2}}{2}\vect{\alpha}^{2}_{n}}}}{\vect{\alpha}^{2}_{n}}
\end{equation}

And the time step update is given by
\begin{equation}
  \begin{split}
    \vect{u}_{n+1} &= \vect{u}_{n}+\Delta t\dot{\vect{u}}_{n} +\dfrac{{\Delta t}^{2}}{2}\vect{\alpha}^{2}_{n} \\
    \dot{\vect{u}}_{n+1} &= \dot{\vect{u}}_{n}+\Delta t\vect{\alpha}^{2}_{n} 
  \end{split}
\end{equation}

\section{Interface Conditions}

\subsection{Variational principles}

The branch of mathematics concerned with the problem of finding a function for
which a certain integral of that function is either at its largest or smallest
value is called the \emph{calculus of variations}. When scientific laws are formulated in terms of the principles of the calculus
of variations they are termed \emph{variational principles}. 

\subsection{Lagrange Multipliers}

