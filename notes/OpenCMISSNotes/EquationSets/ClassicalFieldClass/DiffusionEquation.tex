\subsection{Diffusion Equations} 

\subsection{General Diffusion Equation} 

\subsubsubsection{Governing equations:}

The general form of the diffusion equation with source on a domain $\Omega$ with boundary
$\Gamma$ in \OpenCMISS can be stated as
\begin{equation}
  \boxed{
    \fnof{a}{\vectr{x}}\delby{\fnof{u}{\vectr{x},t}}{t}+
    \divergence{}{\pbrac{\fnof{\tensor{\sigma}}{\vectr{x}}\gradient{}{\fnof{u}{\vectr{x},t}}}}+
    \fnof{b}{\vectr{x},t}=0  
  }
  \label{eqn:diffusionequation}
\end{equation}
where $\vectr{x}\in\Omega$, $\fnof{u}{\vectr{x},t}$ is the quatity that diffuses,
$\fnof{a}{\vectr{x}}$ is a temporal weighting, $\fnof{\tensor{\sigma}}{\vectr{x}}$ is
the conductivity tensor throughout the domain and $\fnof{b}{\vectr{x},t}$ is a
source term.

Appropriate boundary conditions conditions for the diffusion
equation are specification of Dirichlet boundary conditions on the solution,
$d$ \ie
\begin{equation}
  \fnof{u}{\vectr{x},t} = \fnof{d}{\vectr{x},t} \quad \vectr{x}\in\Gamma_{D},
  \label{eqn:diffusionDirichletBC} 
\end{equation}
and/or Neumann conditions in terms of the solution flux in the normal
direction, $e$ \ie
\begin{equation}
  \fnof{q}{\vectr{x}} = \dotprod{\pbrac{\fnof{\tensor{\sigma}}{\vectr{x}}
      \gradient{}{\fnof{u}{\vectr{x}}}}}{\normal} =
  \fnof{e}{\vectr{x}} \quad \vectr{x}\in\Gamma_{N},
  \label{eqn:diffusionNeumannBC} 
\end{equation}
where $\fnof{q}{\vectr{x},t}$ is the flux in the normal direction, $\normal$ is the normal
vector to the boudary and $\Gamma=\union{\Gamma_{D}}{\Gamma_{N}}$.

Appropriate initial conditions for the diffusion equation are the
specification of an initial value of the solution, $f$ \ie
\begin{equation}
  \fnof{u}{\vectr{x},0} = \fnof{f}{\vectr{x}} \quad \vectr{x}\in\Omega.
  \label{eqn:diffusionInitialC} 
\end{equation}

\subsubsubsection{Weak formulation:}

The corresponding weak form of \eqnref{eqn:diffusionequation} can be found by
integrating over the spatial domain with a test function \ie
\begin{equation}
  \gint{\Omega}{}{a\delby{u}{t}+\pbrac{\divergence{}{\pbrac{\tensor{\sigma}
          \gradient{}{u}}}+a}w}{\Omega}=0
  \label{eqn:diffusionweakform1}
\end{equation}
where $w$ is a suitable spatial test function.

Applying the divergence theorem in \eqnref{eqn:DivergenceTheormScalarVector}
to \eqnref{eqn:diffusionweakform1} gives
\begin{equation}
  \gint{\Omega}{}{\pbrac{a\delby{u}{t}}w}{\Omega}-
      \gint{\Omega}{}{\dotprod{\tensor{\sigma}\gradient{}{u}}{\gradient{}{w}}}{\Omega}
      +\gint{\Gamma}{}{\pbrac{\dotprod{\tensor{\sigma}\gradient{}{u}}{\normal}}w}{\Gamma}+
      \gint{\Omega}{}{bw}{\Omega}=0
  \label{eqn:diffusionweakform2}
\end{equation}
where $\normal$ is the unit outward normal vector to the boundary $\Gamma$.

\subsubsubsection{Tensor notation:}

\Eqnref{eqn:diffusionweakform2} can be written in tensor notation as
\begin{equation}
  \gint{\Omega}{}{a\dot{u}w}{\Omega}-
  \gint{\Omega}{}{G^{jk}\sigma^{i}_{j}\covarderiv{u}{i}\covarderiv{w}{k}}{\Omega}+
  \gint{\Gamma}{}{G^{jk}\sigma^{i}_{j}\covarderiv{u}{i}n_{k}w}{\Gamma} +
  \gint{\Omega}{}{bw}{\Omega}=0
  \label{eqn:diffusiontensorform1}
\end{equation}
or
\begin{equation}
  \gint{\Omega}{}{a\dot{u}w}{\Omega}-
  \gint{\Omega}{}{G^{jk}\sigma^{i}_{j}\covarderiv{u}{i}\covarderiv{w}{k}}{\Omega}+
  \gint{\Gamma}{}{qw}{\Gamma} +
  \gint{\Omega}{}{bw}{\Omega}=0
  \label{eqn:diffusiontensorform2}
\end{equation}
where $G^{jk}$ is the contravariant metric tensor for the spatial coordinate system.

\subsubsubsection{Finite element formulation:}

We can now discretise the spatial domain into finite elements \ie $\Omega=
\displaystyle{\bigcup_{e=1}^{E}}\Omega_{e}$ with
$\Gamma=\displaystyle{\bigcup_{f=1}^{F}}\Gamma_{f}$. 
\Eqnref{eqn:diffusiontensorform2} becomes
\begin{equation}
  \dsum_{e=1}^{E}\gint{\Omega_{e}}{}{a\dot{u}w}{\Omega}-
  \dsum_{e=1}^{E}\gint{\Omega_{e}}{}{G^{jk}\sigma^{i}_{j}
    \covarderiv{u}{i}\covarderiv{w}{k}}{\Omega}+
  \dsum_{f=1}^{F}\gint{\Gamma_{f}}{}{qw}{\Gamma} +
  \dsum_{e=1}^{E}\gint{\Omega_{e}}{}{bw}{\Omega}=0
  \label{eqn:diffusionfemform}
\end{equation}

If we now assume that the dependent variable $u$ can be interpolated
separately in space and in time we can write
\begin{equation}
  \fnof{u}{\vectr{x},t}=\gbfn{n}{}{\vectr{x}}\fnof{\nodept{u}{n}}{t}
\end{equation}
or, in standard interpolation notation within an element,
\begin{equation}
  \fnof{u}{\vectr{\xi},t}=\gbfn{n}{\beta}{\vectr{\xi}}
  \fnof{\nodedof{u}{n}{\beta}}{t}\gsf{n}{\beta}
\end{equation}
where $\fnof{\nodedof{u}{n}{\beta}}{t}$ are the time varying nodal
degrees-of-freedom for node $n$, global derivative $\beta$,
$\gbfn{n}{\beta}{\vectr{\xi}}$ are the corresponding basis functions 
and $\gsf{n}{\beta}$ are the scale factors. 

We can also interpolate the other variables in a similar manner \ie
\begin{equation}
  \begin{split}
    \fnof{q}{\vectr{\xi},t} &= \gbfn{o}{\gamma}{\vectr{\xi}}
    \fnof{\nodedof{q}{o}{\gamma}}{t}\gsf{o}{\gamma} \\
    \fnof{a}{\vectr{\xi}} &=\gbfn{p}{\delta}{\vectr{\xi}}
    \nodedof{a}{p}{\delta}\gsf{p}{\delta} \\
    \fnof{\tilde{\tensor{\sigma}}}{\vectr{\xi}}
    &=\gbfn{p}{\delta}{\vectr{\xi}}\nodedof{\tilde{\tensor{\sigma}}}{p}{\delta}
    \gsf{p}{\delta} \\
    \fnof{b}{\vectr{\xi},t} &=
    \gbfn{p}{\delta}{\vectr{\xi}}\fnof{\nodedof{b}{p}{\delta}}{t}\gsf{p}{\delta}
  \end{split}
\end{equation}
where $\fnof{\nodedof{q}{o}{\gamma}}{t}$, $\nodedof{a}{p}{\delta}$,
$\nodedof{\tilde{\tensor{\sigma}}}{p}{\delta}$ and 
$\fnof{\nodedof{b}{p}{\delta}}{t}$ are the
nodal degrees-of-freedom for the variables.

For a Galerkin finite element formulation we also choose the spatial weighting
function $w$ to be equal to the basis fucntions \ie
\begin{equation}
  \fnof{w}{\vectr{\xi}}=\gbfn{m}{\alpha}{\vectr{\xi}}\gsf{m}{\alpha}
\end{equation}

\subsubsubsection{Spatial integration:}

Adopting the standard integration notation we can write the spatial
integration term in \eqnref{eqn:diffusionfemform} as
\begin{multline}
  \dsum_{e=1}^{E}\gint{\vectr{0}}{\vectr{1}}{\fnof{a}{\vectr{\xi}}
    \delby{\pbrac{\gbfn{n}{\beta}{\vectr{\xi}}\fnof{\nodedof{u}{\beta}{n}}{t}
        \gsf{n}{\beta}}}{t}\gbfn{m}{\alpha}{\vectr{\xi}}\gsf{m}{\alpha}
    \abs{\fnof{\matr{J}}{\vectr{\xi}}}}{\vectr{\xi}} - \\
  \dsum_{e=1}^{E}\dintl{\vectr{0}}{\vectr{1}}G^{jk}\delby{x^{i}}{\xi^{d}}\delby{\xi^{e}}{x^{j}}
  \delby{\xi^{d}}{\nu^{a}}\delby{\nu^{b}}{\xi^{e}}\fnof{{\tilde{\sigma}}^{a}_{.b}}{\vectr{\xi}}
  \delby{\xi^{s}}{x^{i}}
  \quad\quad\quad\quad\quad\quad\quad\quad\quad\quad \\ \quad\quad\quad\quad\quad\quad\quad\quad\quad\quad
  \delby{\pbrac{\gbfn{\beta}{n}{\vectr{\xi}}\fnof{\nodedof{u}{\beta}{n}}{t}
      \gsf{\beta}{n}}}{\xi^{s}}\delby{\xi^{r}}{x^{k}}\delby{
    \pbrac{\gbfn{\alpha}{m}{\vectr{\xi}}\gsf{\alpha}{m}}}{\xi^{r}}
  \abs{\fnof{\matr{J}}{\vectr{\xi}}}d\vectr{\xi} + \\
  \dsum_{f=1}^{F}\gint{\vectr{0}}{\vectr{1}}{\gbfn{o}{\gamma}{\vectr{\xi}}
    \fnof{\nodedof{q}{o}{\gamma}}{t}\gsf{o}{\gamma}\gbfn{m}{\alpha}{\vectr{\xi}}
    \gsf{m}{\alpha}\abs{\fnof{\matr{J}}{\vectr{\xi}}}}{\vectr{\xi}} + \\
  \dsum_{e=1}^{E}\gint{\vectr{0}}{\vectr{1}}{\fnof{b}{\vectr{\xi},t}\gbfn{m}{\alpha}{\vectr{\xi}}
    \gsf{m}{\alpha}\abs{\fnof{\matr{J}}{\vectr{\xi}}}}{\vectr{\xi}} = 0
  \label{eqn:diffusionspatialintegration1}
\end{multline}
where $\fnof{\matr{J}}{\vectr{\xi}}$ is the \emph{Jacobian} of the
transformation from the integration $\vectr{x}$ to $\vectr{\xi}$ coordinates.

Taking values that are constant over the integration interval outside the
integration gives
\begin{multline}
  \dsum_{e=1}^{E}\fnof{\nodedof{\dot{u}}{\beta}{n}}{t}\gsf{m}{\alpha}\gsf{n}{\beta}
  \gint{\vectr{0}}{\vectr{1}}{\fnof{a}{\vectr{\xi}}
    \gbfn{m}{\alpha}{\vectr{\xi}}\gbfn{n}{\beta}{\vectr{\xi}}
    \abs{\fnof{\matr{J}}{\vectr{\xi}}}}{\vectr{\xi}} - \\
  \dsum_{e=1}^{E}\fnof{\nodedof{u}{\beta}{n}}{t}\gsf{\alpha}{m}\gsf{\beta}{n}
  \gint{\vectr{0}}{\vectr{1}}{\delby{\gbfn{\alpha}{m}{\vectr{\xi}}}{\xi^{r}}
  \delby{\gbfn{\beta}{n}{\vectr{\xi}}}{\xi^{s}}\fnof{\gamma^{rs}}{\vectr{\xi}}
  \abs{\fnof{\matr{J}}{\vectr{\xi}}}}{\vectr{\xi}} + \\
  \dsum_{f=1}^{F}\fnof{\nodedof{q}{o}{\gamma}}{t}\gsf{m}{\alpha}\gsf{o}{\gamma}
  \gint{\vectr{0}}{\vectr{1}}{\gbfn{m}{\alpha}{\vectr{\xi}}\gbfn{o}{\gamma}{\vectr{\xi}}
    \abs{\fnof{\matr{J}}{\vectr{\xi}}}}{\vectr{\xi}} + \\
  \dsum_{e=1}^{E}\fnof{\nodedof{b}{p}{\delta}}{t}\gsf{m}{\alpha}\gsf{p}{\delta}
  \gint{\vectr{0}}{\vectr{1}}{\gbfn{m}{\alpha}{\vectr{\xi}}\gbfn{p}{\delta}{\vectr{\xi}}
    \abs{\fnof{\matr{J}}{\vectr{\xi}}}}{\vectr{\xi}} = 0
  \label{eqn:diffusionspatialintegration1}
\end{multline}
where $\fnof{\gamma^{rs}}{\vectr{\xi}}$ is defined in 
\eqnthrurefs{eqn:diffusiongammadefinition1}{eqn:diffusiongammadefinition3}.

This is an equation of the form
\begin{equation}
  \matr{C}\fnof{\dot{\vect{u}}}{t}+\matr{K}\fnof{\vect{u}}{t}+\fnof{\vect{f}}{t}=\vect{0}
\end{equation}
where
\begin{equation}
  \fnof{\vect{f}}{t}=\matr{N}\fnof{\vect{q}}{t}+\matr{R}\fnof{\vect{b}}{t}
\end{equation}

The elemental damping matrix, $C^{\alpha\beta}_{mn}$, is given by
\begin{equation}
  C^{\alpha\beta}_{mn} =
  \gsf{m}{\alpha}\gsf{n}{\beta}\gint{\vectr{0}}{\vectr{1}}{\fnof{a}{\vectr{\xi}}
    \gbfn{m}{\alpha}{\vectr{\xi}}\gbfn{n}{\beta}{\vectr{\xi}}
    \abs{\fnof{\matr{J}}{\vectr{\xi}}}}{\vectr{\xi}}
\end{equation}

The elemental stiffness matrix, $K^{\alpha\beta}_{mn}$, is given by
\begin{equation}
  K^{\alpha\beta}_{mn} = -\gsf{m}{\alpha}\gsf{n}{\beta}\gint{\vectr{0}}{\vectr{1}}{
    \delby{\gbfn{\alpha}{m}{\vectr{\xi}}}{\xi^{r}}\delby{\gbfn{\beta}{m}{\vectr{\xi}}}{\xi^{s}}
    \fnof{\gamma^{rs}}{\vectr{\xi}}\abs{\fnof{\matr{J}}{\vectr{\xi}}}}{\vectr{\xi}}
\end{equation}

The elemental flux matrix, $N^{\alpha\gamma}_{mo}$, is given by
\begin{equation}
  N^{\alpha\gamma}_{mo} =\gsf{m}{\alpha}\gsf{o}{\gamma}
  \gint{\vectr{0}}{\vectr{1}}{\gbfn{m}{\alpha}{\vectr{\xi}}\gbfn{o}{\gamma}{\vectr{\xi}}
    \abs{\fnof{\matr{J}}{\vectr{\xi}}}}{\vectr{\xi}}
\end{equation}
and the elemental source matrix, $R^{\alpha\delta}_{mp}$, 
\begin{equation}
  R^{\alpha\delta}_{mp}=\gsf{m}{\alpha}\gsf{p}{\delta}
  \gint{\vectr{0}}{\vectr{1}}{\gbfn{m}{\alpha}{\vectr{\xi}}\gbfn{p}{\delta}{\vectr{\xi}}
    \abs{\fnof{\matr{J}}{\vectr{\xi}}}}{\vectr{\xi}}
\end{equation}
