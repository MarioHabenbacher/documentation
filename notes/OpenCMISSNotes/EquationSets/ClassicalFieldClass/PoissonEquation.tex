\subsection{Poisson Equations} 

\subsubsection{Generalised Poisson Equation} 

\subsubsubsection{Governing equations:}

The general form of Poisson's equation with source on a domain $\Omega$ with boundary
$\Gamma$ in \OpenCMISS can be stated as
\begin{equation}
  \boxed{
    \divergence{}{\pbrac{\fnof{\tensor{\sigma}}{\vect{x}}\gradient{}{\fnof{u}{\vect{x}}}}}+
    \fnof{a}{\vect{x}}=0
  }
  \label{eqn:Poissonequation}
\end{equation}
where $\vect{x}\in\Omega$, $\fnof{u}{\vect{x}}$ is the dependent variable, 
$\fnof{\tensor{\sigma}}{\vect{x}}$ is the conductivity tensor throughout 
the domain and $\fnof{a}{\vect{x}}$ is a source term.

Appropriate boundary conditions conditions for the diffusion
equation are specification of Dirichlet boundary conditions on the solution,
$d$ \ie
\begin{equation}
  \fnof{u}{\vect{x},t} = \fnof{d}{\vect{x},t} \quad \vect{x}\in\Gamma_{D},
  \label{eqn:PoissonDirichletBC} 
\end{equation}
and/or Neumann conditions in terms of the solution flux in the normal
direction, $e$ \ie
\begin{equation}
  \fnof{q}{\vect{x}} = \dotprod{\pbrac{\fnof{\tensor{\sigma}}{\vect{x}}
      \gradient{}{\fnof{u}{\vect{x}}}}}{\normal} =
  \fnof{e}{\vect{x}} \quad \vect{x}\in\Gamma_{N},
  \label{eqn:PoissonNeumannBC} 
\end{equation}
where $\fnof{q}{\vect{x},t}$ is the flux in the normal direction, $\normal$ is the normal
vector to the boudary and $\Gamma=\union{\Gamma_{D}}{\Gamma_{N}}$.

\subsubsection{Weak formulation:}

The corresponding weak form of \eqnref{eqn:Poissonequation} can be found by
integrating over the spatial domain with a test function \ie
\begin{equation}
  \gint{\Omega}{}{\pbrac{\divergence{}{\pbrac{\tensor{\sigma}\gradient{}{u}}}+a}w}{\Omega}=0
  \label{eqn:Poissonweakform1}
\end{equation}
where $w$ is a suitable spatial test function.

Applying the divergence theorem in \eqnref{eqn:DivergenceTheormScalarVector}
to \eqnref{eqn:Poissonweakform1} gives
\begin{equation}
  -\gint{\Omega}{}{\dotprod{\tensor{\sigma}
      \gradient{}{u}}{\gradient{}{w}}}{\Omega}
  +\gint{\Gamma}{}{\pbrac{\dotprod{\tensor{\sigma}\gradient{}{u}}{\normal}}w}{\Gamma}+
  \gint{\Omega}{}{aw}{\Omega}=0
  \label{eqn:Poissonweakform2}
\end{equation}
where $\normal$ is the unit outward normal vector to the boundary $\Gamma$.

\subsubsection{Tensor notation:}

\Eqnref{eqn:Poissonweakform2} can be written in tensor notation as
\begin{equation}
  -\gint{\Omega}{}{G^{jk}\sigma^{i}_{j}\covarderiv{u}{i}\covarderiv{w}{k}}{\Omega}+
  \gint{\Gamma}{}{G^{jk}\sigma^{i}_{j}\covarderiv{u}{i}n_{k}w}{\Gamma} +
  \gint{\Omega}{}{aw}{\Omega}=0
  \label{eqn:Poissontensorform1}
\end{equation}
or
\begin{equation}
  \gint{\Omega}{}{G^{jk}\sigma^{i}_{j}\covarderiv{u}{i}\covarderiv{w}{k}}{\Omega}
  =\gint{\Gamma}{}{qw}{\Gamma} +
  \gint{\Omega}{}{aw}{\Omega}=0
  \label{eqn:Poissontensorform2}
\end{equation}
where $G^{jk}$ is the contravariant metric tensor for the spatial coordinate system.

\subsubsection{Finite element formulation:}


We can now discretise the spatial domain into finite elements \ie $\Omega=
\displaystyle{\bigcup_{e=1}^{E}}\Omega_{e}$ with
$\Gamma=\displaystyle{\bigcup_{f=1}^{F}}\Gamma_{f}$. 
\Eqnref{eqn:Poissontensorform2} becomes
\begin{equation}
  \dsum_{e=1}^{E}\gint{\Omega_{e}}{}{G^{jk}\sigma^{i}_{j}
    \covarderiv{u}{i}\covarderiv{w}{k}}{\Omega}=
  \dsum_{f=1}^{F}\gint{\Gamma_{f}}{}{qw}{\Gamma} +
  \dsum_{e=1}^{E}\gint{\Omega_{e}}{}{aw}{\Omega}
  \label{eqn:Poissonfemform}
\end{equation}

We can now interpolate the variables with basis functions \ie
\begin{equation}
  \begin{split}
    \fnof{u}{\vect{\xi}} &=
    \gbfn{n}{\beta}{\vect{\xi}}\nodedof{u}{n}{\beta}\gsf{n}{\beta} \\
    \fnof{q}{\vect{\xi}} &= 
    \gbfn{o}{\gamma}{\vect{\xi}}\nodedof{q}{o}{\gamma}\gsf{o}{\gamma} \\
    \fnof{\tilde{\tensor{\sigma}}}{\vect{\xi}} &=
    \gbfn{p}{\delta}{\vect{\xi}}\nodedof{\tilde{\tensor{\sigma}}}{p}{\delta}\gsf{p}{\delta} \\
    \fnof{a}{\vect{\xi}} &=
    \gbfn{p}{\delta}{\vect{\xi}}\nodedof{a}{p}{\delta}\gsf{p}{\delta}
  \end{split}
\end{equation}
where $\nodedof{u}{n}{\beta}$, $\nodedof{q}{o}{\gamma}$, 
$\nodedof{\tilde{\tensor{\sigma}}}{p}{\delta}$ and $\nodedof{a}{p}{\delta}$ are the
nodal degrees-of-freedom for the variables.

For a Galerkin finite element formulation we also choose the spatial weighting
function $w$ to be equal to the basis fucntions \ie
\begin{equation}
  \fnof{w}{\vect{\xi}}=\gbfn{m}{\alpha}{\vect{\xi}}\gsf{m}{\alpha}
\end{equation}

\subsubsection{Spatial integration:}

Adopting the standard integration notation we can write the spatial
integration term in \eqnref{eqn:Poissonfemform} as
\begin{multline}
  \dsum_{e=1}^{E}\dintl{\vect{0}}{\vect{1}}G^{jk}\delby{x^{i}}{\xi^{d}}
    \delby{\xi^{e}}{x^{j}}\delby{\xi^{d}}{\nu^{a}}
    \delby{\nu^{b}}{\xi^{e}}\fnof{{\tilde{\sigma}}^{a}_{.b}}{\vect{\xi}}\delby{\xi^{s}}{x^{i}} \\
    \delby{\pbrac{\gbfn{n}{\beta}{\vect{\xi}}\nodedof{u}{n}{\beta}\gsf{n}{\beta}}}{\xi^{s}}
    \delby{\xi^{r}}{x^{k}}\delby{\pbrac{\gbfn{m}{\alpha}{\vect{\xi}}\gsf{m}{\alpha}}}{\xi^{r}}
    \abs{\fnof{\matr{J}}{\vect{\xi}}}d\vect{\xi} \\ 
  = \dsum_{f=1}^{F}\gint{\Gamma_{f}}{}{\gbfn{o}{\gamma}{\vect{\xi}}\nodedof{q}{o}{\gamma}
    \gsf{o}{\gamma}\gbfn{m}{\alpha}{\vect{\xi}}\gsf{m}{\alpha}}{\Gamma} + \\
  \dsum_{e=1}^{E}\gint{\vect{0}}{\vect{1}}{\fnof{a}{\vect{\xi}}\gbfn{m}{\alpha}{\vect{\xi}}
    \gsf{m}{\alpha}\abs{\fnof{\matr{J}}{\vect{\xi}}}}{\vect{\xi}}
\end{multline}
where $\fnof{\matr{J}}{\vect{\xi}}$ is the \emph{Jacobian} of the
transformation from the integration $\vect{x}$ to $\vect{\xi}$ coordinates.

Taking values that are constant over the integration interval outside the
integration gives
\begin{multline}
  \dsum_{e=1}^{E}\nodedof{u}{n}{\beta}\gsf{m}{\alpha}\gsf{n}{\beta}
  \gint{\vect{0}}{\vect{1}}{\delby{\gbfn{m}{\alpha}{\vect{\xi}}}{\xi^{r}}
  \delby{\gbfn{n}{\beta}{\vect{\xi}}}{\xi^{s}}\fnof{\gamma^{rs}}{\vect{\xi}}
  \abs{\fnof{\matr{J}}{\vect{\xi}}}}{\vect{\xi}} \\ 
  =\dsum_{f=1}^{F}\nodedof{q}{o}{\gamma}\gsf{m}{\alpha}\gsf{o}{\gamma}
  \gint{\Gamma_{f}}{}{\gbfn{m}{\alpha}{\vect{\xi}}\gbfn{o}{\gamma}{\vect{\xi}}
  }{\Gamma} + \\
  \dsum_{e=1}^{E}\nodedof{a}{p}{\delta}\gsf{m}{\alpha}\gsf{p}{\delta}
  \gint{\vect{0}}{\vect{1}}{\gbfn{m}{\alpha}{\vect{\xi}}
    \gbfn{p}{\delta}{\vect{\xi}}\abs{\fnof{\matr{J}}{\vect{\xi}}}}{\vect{\xi}}
\end{multline}
where $\fnof{\gamma^{rs}}{\vect{\xi}}$ is defined in 
\eqnthrurefs{eqn:diffusiongammadefinition1}{eqn:diffusiongammadefinition3}.

This is an equation of the form
\begin{equation}
  \matr{K}\vect{u}=\vect{f}
\end{equation}
where
\begin{equation}
  \vect{f}=\matr{N}\vect{q}+\matr{R}\vect{a}
\end{equation}

The elemental stiffness matrix, $K^{\alpha\beta}_{mn}$, is given by
\begin{equation}
  K^{\alpha\beta}_{mn}=\gsf{m}{\alpha}\gsf{m}{\beta}
  \gint{\vect{0}}{\vect{1}}{\delby{\gbfn{m}{\alpha}{\vect{\xi}}}{\xi^{r}}
    \delby{\gbfn{n}{\beta}{\vect{\xi}}}{\xi^{s}}\fnof{\gamma^{rs}}{\vect{\xi}}
    \abs{\fnof{\matr{J}}{\vect{\xi}}}}{\vect{\xi}}
  \label{eqn:elementalfemlhs2}
\end{equation}

The elemental flux matrix, $N^{\alpha\gamma}_{mo}$, is given by
\begin{equation}
  N^{\alpha\gamma}_{mo} =\gsf{m}{\alpha}\gsf{o}{\gamma}
  \gint{\vect{0}}{\vect{1}}{\gbfn{m}{\alpha}{\vect{\xi}}\gbfn{o}{\gamma}{\vect{\xi}}
    \abs{\fnof{\matr{J}}{\vect{\xi}}}}{\vect{\xi}}
\end{equation}
and the elemental source matrix, $R^{\alpha\delta}_{mp}$, 
\begin{equation}
  R^{\alpha\delta}_{mp}=\gsf{m}{\alpha}\gsf{p}{\delta}
  \gint{\vect{0}}{\vect{1}}{\gbfn{m}{\alpha}{\vect{\xi}}\gbfn{p}{\delta}{\vect{\xi}}
    \abs{\fnof{\matr{J}}{\vect{\xi}}}}{\vect{\xi}}
\end{equation}

