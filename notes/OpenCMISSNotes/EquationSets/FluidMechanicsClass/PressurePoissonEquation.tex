\subsection{Pressure Poisson Equation}

The Navier-Stokes equation can be written as
\begin{equation}
  \rho\pbrac{\delby{\vect{u}}{t}-\dotprod{\vect{u}}{\gradient{\vect{u}}}}=-\grad{p}+\mu\diverg{\gradient{\vect{u}}}
  \label{eqn:NavierStokes}
\end{equation}

Rearranging for $\grad{p}$ gives
\begin{equation}
  \grad{p}=\mu\diverg{\gradient{\vect{u}}}-\rho\pbrac{\delby{\vect{u}}{t}-\dotprod{\vect{u}}{\gradient{\vect{u}}}}  
\end{equation}
or
\begin{equation}
  \gradient{p}=\vect{b}
  \label{eqn:NavierStokesPressureForm}
\end{equation}
where
\begin{equation}
  \vect{b}=\mu\diverg{\gradient{\vect{u}}}-\rho\pbrac{\delby{\vect{u}}{t}-\dotprod{\vect{u}}{\gradient{\vect{u}}}}
\end{equation}

Note that from \eqnref{eqn:NavierStokesPressureForm} we can write
\begin{equation} 
  \dotprod{\gradient{p}}{\normal}=\dotprod{\vect{b}}{\normal}
  \label{eqn:NavierStokesPressureFormNormal}
\end{equation}

Taking the divergence of both sides of \eqnref{eqn:NavierStokesPressureForm} gives
\begin{equation}
  \diverg{\gradient{p}}=\diverg{\vect{b}}
  \label{eqn:PressurePoisson}
\end{equation}

The corresponding weak form of \eqnref{eqn:PressurePoisson} is
\begin{equation}
  \gint{\Omega}{}{\diverg{\gradient{p}}w}{\Omega}=\gint{\Omega}{}{\diverg{\vect{b}}w}{\Omega}
  \label{eqn:PressurePoissonWeakForm}
\end{equation}

Now the divergence theorm states that
\begin{equation}
  \gint{\Omega}{}{\diverg{\tensor{F}}}{\Omega}=\gint{\Gamma}{}{\dotprod{\tensor{F}}{\normal}}{\Gamma}
  \label{eqn:DivergenceTheorm}
\end{equation}

Applying the divergence theorm where $\tensor{F}=g\vect{f}$ gives
\begin{equation}
  \gint{\Omega}{}{\pbrac{\dotprod{\vect{f}}{\gradient{g}}+g\diverg{\vect{f}}}}{\Omega}=\gint{\Gamma}{}{g\dotprod{\vect{f}}{\normal}}{\Gamma}
  \label{eqn:DivergenceTheormScalarVector}
\end{equation}

Note that when $\vect{f}=\gradient{f}$ is used in
\eqnref{eqn:DivergenceTheormScalarVector} you get Green's identity \ie
\begin{equation}
  \gint{\Omega}{}{\pbrac{\dotprod{\gradient{f}}{\gradient{g}}+g\diverg{\gradient{f}}}}{\Omega}=\gint{\Gamma}{}{g\dotprod{\gradient{f}}{\normal}}{\Gamma}
  \label{eqn:GreensIdentity1}
\end{equation}

Now if we apply \eqnref{eqn:GreensIdentity1} to the left hand side of
\eqnref{eqn:PressurePoissonWeakForm} we get
\begin{equation}
  \gint{\Omega}{}{\diverg{\gradient{p}}w}{\Omega}=\gint{\Gamma}{}{\dotprod{\gradient{p}}{\normal}w}{\Gamma}-
  \gint{\Omega}{}{\dotprod{\gradient{p}}{\gradient{w}}}{\Omega}
\end{equation}
or from \eqnref{eqn:NavierStokesPressureFormNormal}
\begin{equation}
  \gint{\Omega}{}{\diverg{\gradient{p}}w}{\Omega}=\gint{\Gamma}{}{\dotprod{\vect{b}}{\normal}w}{\Gamma}-
  \gint{\Omega}{}{\dotprod{\gradient{p}}{\gradient{w}}}{\Omega}
  \label{eqn:PressurePoissonWeakFormLHS}
\end{equation}

Now if we apply \eqnref{eqn:DivergenceTheormScalarVector} to the right hand
side of \eqnref{eqn:PressurePoissonWeakForm} we get
\begin{equation}
  \gint{\Omega}{}{\diverg{\vect{b}}w}{\Omega}=\gint{\Gamma}{}{\dotprod{\vect{b}}{\normal}w}{\Gamma}-
  \gint{\Omega}{}{\dotprod{\vect{b}}{\gradient{w}}}{\Omega}
  \label{eqn:PressurePoissonWeakFormRHS}
\end{equation}

Substituting \eqnref{eqn:PressurePoissonWeakFormLHS} and
\eqnref{eqn:PressurePoissonWeakFormRHS} into
\eqnref{eqn:PressurePoissonWeakForm} gives
\begin{equation}
  \gint{\Gamma}{}{\dotprod{\vect{b}}{\normal}w}{\Gamma}-
  \gint{\Omega}{}{\dotprod{\gradient{p}}{\gradient{w}}}{\Omega}=\gint{\Gamma}{}{\dotprod{\vect{b}}{\normal}w}{\Gamma}-
  \gint{\Omega}{}{\dotprod{\vect{b}}{\gradient{w}}}{\Omega}
\end{equation} 
or
\begin{equation}
  \gint{\Omega}{}{\dotprod{\gradient{p}}{\gradient{w}}}{\Omega}=\gint{\Omega}{}{\dotprod{\vect{b}}{\gradient{w}}}{\Omega}
  \label{eqn:PressurePoissonFinalWeakForm}
\end{equation}

Now, using a standard Galerkin Finite Element approach
\eqnref{eqn:PressurePoissonFinalWeakForm} can be written in the following form
 \begin{equation}
   \matr{K}_{p}\vect{p}=\vect{s}
  \label{eqn:PressurePoissonFinalWeakForm}
\end{equation}
where $\matr{K}_{p}$ is the pressure stiffness matrix, $\vect{p}$ the vector of unknown
pressure DOFs and $\vect{s}$ the source vector. 

Note that you can write
$\vect{s}$ in terms of $\matr{K}_{u}$ the velocity
stiffness matrix, $\matr{M}$ the mass matrix, $\vect{u}$ the vector of know
velocity DOFs and $\fnof{\vect{h}}{\vect{u}}$ the nonlinear Navier-Stokes
vector.

Note that $\matr{K}_{P}$ is most likely signular and therefore you will need to
set a reference pressure Dirichlet condition somewhere and measure the
pressures relative to this boundary condition value.

Note that this formulation removes all the nasty Neumann conditions in favour
of a Dirichlet condition. It also removes the surface integrals in favour of
volume integrals!

\newpage 
