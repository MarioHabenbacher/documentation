\subsection{Stokes Equations}

\subsubsection{Governing equations:}

Stokes flow is a type of fluid flow where advective inertial forces are small
compared with viscous forces. The Reynolds number is low, i.e.
$\textit{Re}\ll 1$. For this type of flow, the inertial forces are assumed to
be negligible and the Navier-Stokes equations simplify to give the Stokes
equations. In the common case of an incompressible Newtonian fluid, the linear
Stokes equations (three-dimensional, quasi-static) can be written as:
\begin{equation}
    \vectr{f}-\gradient{}{p}+\mu\laplacian{}{\vectr{u}}=\vectr{0}
  \label{eqn:Stokesequation1}
\end{equation}
accompanied by the conservation of mass
\begin{equation}
  \divergence{}{\vectr{u}}=0
  \label{eqn:Stokesmasequation}
\end{equation}
where $\vectr{u}(\vectr{x},t)=\transpose{\sqbrac{u_1,u_2,u_3}}$ is the
velocity vector depending on spatial coordinates $\vectr{x}=\transpose{\sqbrac{x_1,x_2,x_3}}$
and the time $t$, $p$ is the scalar pressure, $\vectr{f}$ an applied body
force, and the material parameter $\mu$ is fluid viscosity,
respectively. Whereas \eqnref{eqn:Stokesequation1} has been formulated in
Eulerian form, moving domain approaches often require the ALE modification
taking an additional term into account, depending on the fluid density $\rho$
and a correction velocity $\vectr{u}^*$ which yields to:
\begin{equation}
  \vectr{f}-\gradient{}{p}+\mu\laplacian{}{\vectr{u}}=-\rho\dotprod{\vectr{u}^*}{\gradient{}{\vectr{u}}}
  \label{eqn:StokesequationALE}
\end{equation}
Although by definition a Stokes flow has no dependence on time (other than
through changing boundary conditions), \eqnref{eqn:StokesequationALE} can be
modified easily towards a dynamic Stokes flow:
\begin{equation}
    \rho\delby{\vectr{u}}{t}-\rho\dotprod{\vectr{u}^*}{\gradient{}{\vectr{u}}}=\vectr{f}-\gradient{}{p}+\mu\laplacian{}{\vectr{u}}
  \label{eqn:Stokesequation2}
\end{equation}
The following section, however, describes the reordered quasi-static
formulation of \eqnref{eqn:StokesequationALE}:
\begin{equation}
    -\gradient{}{p}+\mu\laplacian{}{\vectr{u}}-\rho\dotprod{\vectr{u}^*}{\gradient{}{\vectr{u}}}=\vectr{f}
  \label{eqn:StokesequationALE2}
\end{equation}

\subsubsection{Weak formulation:}

The corresponding weak form of the equation system consisting of
\eqnref{eqn:Stokesequation1} and \eqnref{eqn:Stokesmasequation} can be written
as:
\begin{equation}
  -\gint{\Omega}{}{\gradient{}{p}\vectr{v}}{\Omega}
  +\gint{\Omega}{}{\mu\laplacian{}{\vectr{u}}\vectr{v}}{\Omega}
  -\gint{\Omega}{}{\rho\dotprod{\vectr{u}^*}{\gradient{}{\vectr{u}}}\vectr{v} }{\Omega}
  +\gint{\Omega}{}{\divergence{}{\vectr{u}}q}{\Omega}=
  \gint{\Omega}{}{\vectr{f}\vectr{v}}{\Omega}  
  \label{eqn:Stokesweakform}
\end{equation}
Applying Green's theorm to \eqnref{eqn:Stokesweakform} gives
\begin{equation}
  \begin{split}
  \gint{\Omega}{}{\divergence{}{\vectr{v}}p}{\Omega}
  -\gint{\Omega}{}{\mu\doubledotprod{\gradient{}{\vectr{v}}}{\gradient{\vectr{u}}}}{\Omega}
  -\gint{\Omega}{}{\rho\dotprod{\vectr{u}^*}{\gradient{}{\vectr{u}}}\vectr{v}}{\Omega}
  +\gint{\Omega}{}{\divergence{}{\vectr{u}}q}{\Omega}=\\
  \gint{\Omega}{}{\vectr{f}\vectr{v}}{\Omega}  
  +\gint{\Gamma}{}{\dotprod{p}{\vectr{n}\vectr{v}}}{\Gamma}
  -\gint{\Gamma}{}{\dotprod{\vectr{v}\gradient{}{\vectr{u}}}{\vectr{n}}}{\Gamma}
  \end{split}
  \label{eqn:Stokesweakform2}
\end{equation}
The left-hand side of \eqnref{eqn:Stokesweakform2} represents the unknown
velocity and pressure field on $\Omega$ whereas the right-hand side represents
Neumann boundary conditions on $\Gamma$ and source terms (here in form of
volume or body forces) on $\Omega$. For now we want to neglect the right-hand
side and continue with the homogeneous form of \eqnref{eqn:Stokesweakform2}
\begin{equation}
  \begin{split}
  \gint{\Omega}{}{\grad\cdot\vectr{v}p}{\Omega}
  -\gint{\Omega}{}{\mu\grad\vectr{v}:\grad\vectr{u}}{\Omega}
  -\gint{\Omega}{}{\rho(\vectr{u}^*\cdot\grad)\vectr{u}\vectr{v}}{\Omega}
  +\gint{\Omega}{}{\grad\cdot \vectr{u}q}{\Omega}=0
  \end{split}
  \label{eqn:Stokesweakform3}
\end{equation}

\subsubsection{Tensor notation:}
If we now consider the integrand of the left hand side of
\eqnref{eqn:Stokesweakform3} in tensorial form with covariant
derivatives then
% \begin{equation}
%   \tensor{\sigma}\grad\phi = \sigma^{i}_{j}\covarderiv{\phi}{i}
% \end{equation}
% and
% \begin{equation}
%   \grad w = \covarderiv{w}{i}
% \end{equation}
% 
% Now, both the above equations represent vectors that are covariant and
% therefore we must convert the first equation to a contravariant vector by 
% multiplying by the contravariant metric tensor (from \vectr{i} to \vectr{x} 
% coordinates) so that we can take the dot product. The final tensorial form is
% \begin{equation}
%   \dotprod{\pbrac{\tensor{\sigma}\grad\phi}}{\grad w} = G^{jk}\sigma^{i}_{j}\covarderiv{\phi}{i}\covarderiv{w}{k}
% \end{equation}
% and thus \eqnref{eqn:Laplaceweightedresidualform} becomes
% \begin{equation}
%   \gint{\Omega}{}{G^{jk}\sigma^{i}_{j}\covarderiv{\phi}{i}\covarderiv{w}{k}}{\Omega}=
%   \gint{\Gamma}{}{\dotprod{\pbrac{\tensor{\sigma}\grad\phi}}{\vectr{n}w}}{\Gamma}
%   \label{eqn:Laplaceweightedresidualtensorform}
% \end{equation}
% 
% \subsubsection{Finite element formulation:}
% We can now discretise the domain into finite elements \ie $\Omega=
% \displaystyle{\bigcup_{e=1}^{E}}\Omega_{e}$, the left hand side of
% \eqnref{eqn:Stokesweakform3} becomes
% \begin{equation}
%   \dsum_{e=1}^{E}\gint{\Omega_{e}}{}{G^{jk}\sigma^{i}_{j}\covarderiv{\phi}{i}\covarderiv{w}{k}}{\Omega}
% \end{equation}
% 
% The next step is to approximate the dependent variable, $\phi$, using basis
% functions. To simplify this for different types of basis functions an
% \emph{interpolation notation} is adopted. This
% interpolation is such that $\gbfn{\alpha}{u}{\vectr{\xi}}$ are the appropriate
% basis functions for the type of element (\eg \bicubicherm, Hermite-sector,
% \etc) and dimension of the problem (one, two or \threedal). For example if
% $\vectr{\xi}=\bbrac{\xi}$ and the element is a cubic Hermite element then
% $\gbfn{\alpha}{u}{\vectr{\xi}}$ are the cubic Hermite basis functions where
% $\alpha$ ranges from $1$ to $2$ and $u$ ranges from $0$ to $1$. If, however,
% $\vectr{\xi}=\bbrac{\xione,\xitwo}$ and the element is a \bicubicherm element
% then $\alpha$ ranges from $1$ to $4$, $u$ ranges from $0$ to $3$ and
% $\gbfn{\alpha}{u}{\vectr{\xi}}$ is the appropriate basis function for the nodal
% variable $\nodedof{\phi}{\alpha}{u}$. The nodal variables are defined as
% follows: $\nodedof{\phi}{\alpha}{0}=\nodept{\phi}{\alpha}$,
% $\nodedof{\phi}{\alpha}{1}=\delby{\nodept{\phi}{\alpha}}{s_{1}}$,
% $\nodedof{\phi}{\alpha}{2}=\delby{\nodept{\phi}{\alpha}}{s_{2}}$,
% $\nodedof{\phi}{\alpha}{3}=\deltwoby{\nodept{\phi}{\alpha}}{s_{1}}{s_{2}}$,
% \etc Hence for the \bicubicherm element the interpolation function
% $\gbfn{2}{3}{\vectr{\xi}}$ multiplies the nodal variable
% $\phi^{2}_{,3}=\deltwoby{\phi^{2}}{s_{1}}{s_{2}}$ and thus the
% interpolation function is $\chbfn{2}{1}{\xione}\chbfn{1}{1}{\xitwo}$.  The
% scale factors for the Hermite interpolation are handled by the introduction of
% an interpolation scale factor $\gsf{\alpha}{u}$ defined as follows:
% $\gsf{\alpha}{0}=1$, $\gsf{\alpha}{1}=\nsftwo{\alpha}{1}$, $\gsf{\alpha}{2}=
% \nsftwo{\alpha}{2}$, $\gsf{\alpha}{3}=\nsftwo{\alpha}{1}\nsftwo{\alpha}{2}$,
% \etc For Lagrangian basis functions the interpolation scale factors are all
% one. The general form of the interpolation notation for $\phi$ is then
% \begin{equation}
%   \fnof{\phi}{\vectr{\xi}}=\gbfn{\alpha}{u}{\vectr{\xi}}\nodedof{\phi}
%   {\alpha}{u}\gsf{(\alpha)}{(u)}
%   \label{eqn:phiinterpolation}
% \end{equation}
% 
% \subsubsection{Spatio-temporal integration:}
% When dealing with integrals a similar interpolation notation is adopted in
% that $d\vectr{\xi}$ is the appropriate infinitesimal for the dimension of the
% problem. The limits of the integral are also written in bold font and indicate
% the appropriate number of integrals for the dimension of the problem.  For
% example if $\vectr{\xi}=\bbrac{\xione,\xitwo}$ then
% $\gint{\vectr{0}}{\vectr{1}}{x}
% {\vectr{\xi}}=\giint{0}{1}{0}{1}{x}{\xione}{\xitwo}$, but if $\vectr{\xi}=
% \bbrac{\xione,\xitwo,\xithree}$ then $\gint{\vectr{0}}{\vectr{1}}{x}
% {\vectr{\xi}}=\dintl{0}{1}\!\dintl{0}{1}\!\dintl{0}{1}\,x\,d\xione d\xitwo
% d\xithree$.
% 
% In order to integrate in $\vectr{\xi}$ coordinates we must convert the spatial
% derivatives to derivatives with respect to $\vectr{\xi}$. Using the tensor
% transformation equations for a covariant vector we have
% \begin{equation}  
%   \covarderiv{\phi}{i}=\delby{\xi^{r}}{x^{i}}\covarderiv{\phi}{r}=\delby{\xi^{r}}{x^{i}}\delby{\phi}{\xi^{r}}
% \end{equation}
% and 
% \begin{equation}
%   \covarderiv{w}{k}=\delby{\xi^{s}}{x^{k}}\covarderiv{w}{s}=\delby{\xi^{s}}{x^{k}}\delby{w}{\xi^{s}}
% \end{equation}
% 
% Now as we only know $\tensor{\sigma}^{*}$, the conductivity in the
% $\vectr{\nu}$ (fibre) coordinate system rather than $\tensor{\sigma}$, the
% conductivity in the $\vectr{x}$ (geometric) coordinate system we must transform the mixed
% tensor ${\sigma^{*}}^{a}_{.b}$ from $\vectr{\nu}$ to $\vectr{x}$ coordinates. However, as the
% fibre coordinate system is defined in terms of $\vectr{\xi}$ coordinates we
% first transform the conductivity tensor from $\vectr{\nu}$ to $\vectr{\xi}$
% coordinates \ie
% \begin{equation}
%   \sigma^{p}_{.q}=\delby{\xi^{p}}{\nu^{a}}\delby{\nu^{b}}{\xi^{q}}{\sigma^{*}}^{a}_{.b}
% \end{equation}
% and then transform the conductivity form $\vectr{\xi}$ coordinates to
% $\vectr{x}$ coordinates \ie
% \begin{equation}
%   \begin{split}
%     \sigma^{i}_{.j} &= \delby{x^{i}}{\xi^{p}}\delby{\xi^{q}}{x^{j}}{\sigma}^{p}_{.q} \\
%     &= \delby{x^{i}}{\xi^{p}}\delby{\xi^{q}}{x^{j}}\delby{\xi^{p}} 
%     {\nu^{a}}\delby{\nu^{b}}{\xi^{q}}{\sigma^{*}}^{a}_{.b}
%   \end{split}
% \end{equation}
% 
% Now, using an interpolated dependent variable and a Galerkin finite element
% approach (where the weighting functions are chosen to be the interpolating
% basis functions \ie $w=\gbfn{\beta}{v}{\vectr{\xi}}\gsf{(\beta)}{(v)}$) yields
% \begin{equation}
%   \dsum_{e=1}^{E}\gint{\vectr{0}}{\vectr{1}}{G^{jk}\delby{x^{i}}{\xi^{p}}
%     \delby{\xi^{q}}{x^{j}}\delby{\xi^{p}} {\nu^{a}}
%     \delby{\nu^{b}}{\xi^{q}}{\sigma^{*}}^{a}_{.b}\delby{\xi^{r}}{x^{i}}
%     \delby{\pbrac{\gbfn{\alpha}{u}{\vectr{\xi}}\nodedof{\phi}{\alpha}{u}\gsf{(\alpha)}{(u)}}}{\xi^{r}}
%     \delby{\xi^{s}}{x^{k}}\delby{\pbrac{\gbfn{\beta}{v}{\vectr{\xi}}\gsf{(\beta)}{(v)}}}{\xi^{s}}
%     \abs{\fnof{\matr{J}}{\vectr{\xi}}}}{\vectr{\xi}}
% \end{equation}
% where $\fnof{\matr{J}}{\vectr{\xi}}$ is the \emph{Jacobian} of the
% transformation from the integration $\vectr{x}$ to $\vectr{\xi}$ coordinates.
% 
% Taking the fixed nodal degrees-of-freedom and scale factors outside the integral we have
% \begin{equation}
%   \dsum_{e=1}^{E}\nodedof{\phi}{\alpha}{u}\gsf{(\alpha)}{(u)}\gsf{(\beta)}{(v)}
%   \gint{\vectr{0}}{\vectr{1}}{G^{jk}\delby{x^{i}}{\xi^{p}}
%     \delby{\xi^{q}}{x^{j}}\delby{\xi^{p}} {\nu^{a}}
%     \delby{\nu^{b}}{\xi^{q}}{\sigma^{*}}^{a}_{.b}\delby{\xi^{r}}{x^{i}}
%     \delby{\gbfn{\alpha}{u}{\vectr{\xi}}}{\xi^{r}}
%     \delby{\xi^{s}}{x^{k}}\delby{\gbfn{\beta}{v}{\vectr{\xi}}}{\xi^{s}}
%     \abs{\fnof{\matr{J}}{\vectr{\xi}}}}{\vectr{\xi}}
%   \label{eqn:elementalfemlhs1}
% \end{equation}
% or
% \begin{equation}
%   \dsum_{e=1}^{E}\nodedof{\phi}{\alpha}{u}\gsf{(\alpha)}{(u)}\gsf{(\beta)}{(v)}
%   \gint{\vectr{0}}{\vectr{1}}{\delby{\gbfn{\alpha}{u}{\vectr{\xi}}}{\xi^{r}}\delby{\gbfn{\beta}{v}{\vectr{\xi}}}{\xi^{s}}\gamma^{rs}
%     \abs{\fnof{\matr{J}}{\vectr{\xi}}}}{\vectr{\xi}}
%   \label{eqn:elementalfemlhs2}
% \end{equation}
% where
% \begin{equation}
%   \gamma^{rs}=G^{jk}\delby{x^{i}}{\xi^{p}}
%     \delby{\xi^{q}}{x^{j}}\delby{\xi^{p}} {\nu^{a}}
%     \delby{\nu^{b}}{\xi^{q}}{\sigma^{*}}^{a}_{.b}\delby{\xi^{r}}{x^{i}}\delby{\xi^{s}}{x^{k}}
% \end{equation}
% 
% Note that for Laplace's equation with no conductivity or fibre fields then we have
% \begin{equation}
%   \gamma^{rs}=G^{ik}\delby{\xi^{r}}{x^{i}}\delby{\xi^{s}}{x^{k}}
% \end{equation}
% and that for rectangular-Cartesian coordinates systems
% $G^{ik}=\kronecker{i}{k}$ and thus $i=k$. This now gives
% \begin{equation}
%   \gamma^{rs}=\delby{\xi^{r}}{x^{i}}\delby{\xi^{s}}{x^{i}}=g^{rs}
%   \label{eqn:contrametrictensor}
% \end{equation}
% where $g^{rs}$ is the \emph{contravariant metric tensor} from $\vectr{x}$ to
% $\vectr{\xi}$ coordinates. It may thus be helpful to think about $\gamma^{rs}$
% as a scaled/transformed contravariant metric tensor.
% 
% \subsubsection{Coded OpenCMISS formulation:}
% Finally, we use the Gaussian quadrature rule, usually stated as a weighted sum of function values at specified Gauss points within the domain of integration. An $n$-point Gaussian quadrature rule yields an exact result for polynomials of degree $2n-1$ or less by a suitable choice of the points $x_i$ and weights $g_i$ for $i = 1,...,n$. Consequently, the formulation implemented can be derived from Equation (\ref{eqn:elementalfemlhs2}) as: 
% \begin{equation}
%   \boxed{
%   \dsum_{e=1}^{E}\nodedof{\phi}{\alpha}{u}\gsf{(\alpha)}{(u)}
%   \dsum_{i=1}^{n}
%   \left(\delby{\gbfn{\alpha}{u}{\vectr{\xi}}}{\xi^{r}}
%   \delby{\gbfn{\beta}{v}{\vectr{\xi}}}{\xi^{s}}\gamma^{rs}\abs{\fnof{\matr{J}}{\vectr{\xi}}}
%   \right)(x_i)g_i
%   }
% \end{equation}
