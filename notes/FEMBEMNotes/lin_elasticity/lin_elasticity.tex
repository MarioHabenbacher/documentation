\chapter{Linear Elasticity} 
\label{cha:linearelasticity}

\section{Introduction}
\label{sec:Intro}

To analyse the stress in various elastic bodies we calculate the strain energy
of the body in terms of nodal displacements and then minimize the strain
energy with respect to these parameters - a technique known as the
\index{Rayleigh-Ritz method}Rayleigh-Ritz. In fact, as we will show later, this leads to
the same algebraic equations as would be obtained by the Galerkin method (now
equivalent to virtual work) but the physical assumptions made (in neglecting
certain strain energy terms) are exposed more clearly in the Rayleigh-Ritz
method. We will first consider one-dimensional truss and beam elements, then
two-dimensional plane stress and plane strain elements, and finally
three-dimensional elasticity.

In all cases the steps are:
\begin{enumerate}
  \item Evaluate the components of strain in terms of nodal displacements,
  \item Evaluate the components of stress from strain using the elastic 
   material constants,
  \item  Evaluate the strain energy for each element by integrating the 
   products of stress and strain components over the element volume,
  \item  Evaluate the potential energy from the sum of total strain energy for
   all elements together with the work done by applied boundary forces, 
  \item  Apply the boundary conditions, \eg by fixing nodal displacements,
  \item  Minimize the potential energy with respect to the unconstrained nodal
   displacements,
  \item  Solve the resulting system of equations for the unconstrained nodal 
   displacements,
  \item  Evaluate the stresses and strains using the nodal displacements and 
   element basis functions,
  \item  Evaluate the boundary reaction forces (or moments) at the nodes where
   displacement is constrained. 
\end{enumerate}

\section{Truss Elements}

\index{Truss elements|(}
Consider the one-dimensional truss of undeformed length $L$ in
\Figref{fig:unitf} with end points $\pbrac{0,0}$ and $\pbrac{x,y}$ and making 
an angle $\theta$ with the x-axis. Under the action of forces in the $x$- and
$y$- directions the right hand end of the truss displaces by $u$ in the 
$x$-direction and $v$ in the $y$-direction, relative to the left hand end.

\begin{figure}[htbp] \centering
  \input{figs/lin_elasticity/truss.pstex}
  \caption{A truss of initial length $L$ is stretched to a new length $l$.   
    Displacements of the right hand end relative to the left hand end are $u$ 
    and $v$ in the $x$- and $y$- directions, respectively.}
  \label{fig:truss}
\end{figure}
                      
The new length is $l$ with axial strain
\begin{align*}
  e=\dfrac{l}{L}-1 &= \dfrac{\sqrt{\pbrac{X+u}^{2} +
      \pbrac{Y+v}^{2}}}{\sqrt{X^{2} + Y^{2}}} -1 \\ 
&=\dfrac{\sqrt{L^{2}+2\pbrac{Xu+Yv}+u^{2}+v^{2}}}{L} - 1 \\ 
&= \sqrt{1+2 \pbrac{\cos \theta . \dfrac{u}{L} + \sin \theta . \dfrac{v}{L}} +
    \dfrac{u^{2}+v^{2}}{L^{2}}} - 1
\end{align*}
using $\dfrac{X}{L}=\cos \theta$ and $\dfrac{Y}{L}=\sin \theta$, where
$\theta$ is defined to be positive in the anticlockwise direction. Neglecting
second order terms in the binomial expansion $\sqrt{\pbrac{1+\varepsilon}} = 1 +
\dfrac{1}{2}\varepsilon + \orderof{\varepsilon^{2}}$, the strain for small
displacements $u$ and $v$ is
\begin{equation}
  e \cong \cos \theta .\dfrac{u}{L} + \sin \theta .\dfrac{v}{L}
  \label{eqn:strain}
\end{equation}

The strain energy associated with this uniaxial stretch is 
\begin{equation}
  \text{SE} = \dfrac{1}{2} \gint{}{}{\sigma e}{V} = \dfrac{1}{2} A \gint{0}{L}
  {\sigma e}{x} = \dfrac{1}{2} \gint{0}{L}{EAe^{2}}{x} = \dfrac{1}{2}  ALEe^{2}
  \label{eqn:sen}
\end{equation}
where $\sigma=Ee$ is the stress in the truss (of cross-sectional area $A$),
linearly related to the strain $e$ via Young's modulus $E$. We now substitute
for $e$ from \eqnref{eqn:strain} into \eqnref{eqn:sen} and put $u=u_{2}-u_{1}$
and $v=v_{2}-v_{1}$, where $\pbrac{u_{1},v_{1}}$ and $\pbrac{u_{2},v_{2}}$ are
the nodal displacements of the two ends of the truss
\begin{equation}
  \text{SE}= \dfrac{1}{2} ALE \pbrac{\cos \theta . \dfrac{u_{2}-u_{1}}{L} + 
    \sin \theta . \dfrac{v_{2}-v_{1}}{L}}^{2}
  \label{eqn:tet}
\end{equation}

The potential energy is the combined strain energy from all trusses in the
structure minus the work done on the structure by external forces. The
Rayleigh-Ritz \index{Rayleigh-Ritz method} approach is to minimize this
potential energy with respect to the nodal displacements once all displacement
boundary conditions have been applied.

For example, consider the system of three trusses shown in
\Figref{fig:system}. A force of \nunit{100}{\kN} is applied in the $x$-direction 
at node $1$.  Node $2$ is a sliding joint and has zero displacement in the
y-direction only.  Node $3$ is a pivot and therefore has zero displacement in
both $x$- and $y$- directions. The problem is to find all nodal displacements and
the stress in the three trusses.

\begin{figure}[htbp] \centering
  \input{figs/lin_elasticity/system.pstex}
  \caption{A system of three trusses.}
  \label{fig:system}
\end{figure}

The strain in truss $1$ (joining nodes $1$ and $3$) is 
\begin{displaymath}
  \dfrac{u_{1}}{L}\cos 30+\dfrac{v_{1}}{L} \sin 30=
  \dfrac{\sqrt{3}}{2}\dfrac{u_{1}}{L}+\dfrac{1}{2}\dfrac{v_{1}}{L}
\end{displaymath}
The strain in truss $2$ (joining nodes $1$ and $2$) is 
\begin{displaymath}
  \dfrac{\pbrac{u_{1}-u_{2}}}{L}\cos 90+\dfrac{v_{1}}{L}\sin 90=\dfrac{v_{1}}{L}
\end{displaymath}
The strain in truss $3$ (joining nodes $2$ and $3$) is 
\begin{displaymath}
  \frac{u_{2}}{L}\cos \pbrac{-30}=\dfrac{\sqrt{3}}{2}\dfrac{u_{2}}{L}
\end{displaymath} 

Since a force of \nunit{100}{\kN} acts at node $1$ in the $x$-direction, 
the potential energy is
\begin{displaymath}  
  \text{PE} =\sum_{\text{trusses}} \dfrac{1}{2} ALEe^{2} - 100 u_{1} =
  \dfrac{1}{2}\frac{AE}{L} \sqbrac{\pbrac{\dfrac{\sqrt{3}}{2} u_{1} +
      \dfrac{1}{2} v_{1}}^{2}+\pbrac{\dfrac{\sqrt{3}}{2} u_{2}}^2 +
      \pbrac{v_{1}}^{2}}- 100u_{1}
\end{displaymath}
[Note that if the force was applied in the negative $x$-direction, the final
  term would be $+ 100 u_{1}$]

Minimizing the potential energy with respect to the three unknowns $u_{1}$,
$v_{1}$ and $u_{2}$  gives
\begin{equation}
  \delby{\text{PE}}{u_{1}} = \dfrac{AE}{L}\pbrac{\dfrac{\sqrt{3}}{2} 
    u_{1} + \dfrac{1}{2} v_{1}}\dfrac{\sqrt{3}}{2} - 100=0
  \label{eqn:poten1}
\end{equation}
\begin{equation}
  \delby{\text{PE}}{v_{1}}  =  \dfrac{AE}{L}\sqbrac{\pbrac{
    \dfrac{\sqrt{3}}{2} u_{1} + \dfrac{1}{2} v_{1}}\dfrac{1}{2} +v_{1}}=0
  \label{eqn:poten2}
\end{equation}
\begin{equation}
  \delby{\text{PE}}{u_{2}} =  \dfrac{AE}{L}\left( \dfrac{\sqrt{3}}{2} 
    u_{2}\right)\dfrac{\sqrt{3}}{2}=0
  \label{eqn:poten3}
\end{equation}
  
If we choose $A =$ \nunit{5\times\tento{-3}}{\mtwo}, $E =$ \nunit{10}{\GPa} and
 $L = $ \nunit{1}{\m} (\eg \nunit{100}{\mm} $\times$ \nunit{50}{\mm} timber truss) 
 then $\dfrac{AE}{L} =$ \nunit{5 \times \tento{-3}}{\mtwo} 
 \nunit{\times \tento{7}}{\kPa /\m} $ = $ \nunit{5 \times \tento{4}}{\kNpm}.

\Eqnref{eqn:poten3} gives 
\begin{displaymath}  
  u_{2} = 0
\end{displaymath} 
\Eqnref{eqn:poten1} gives 
\begin{displaymath}
  3u_{1}+\sqrt{3}v_{1}=4\times\tento{2}/\pbrac{5\times\tento{4}}
\end{displaymath}
\Eqnref{eqn:poten2} gives for two dimensions
\begin{displaymath}
  v_{1} = -\dfrac{\sqrt{3}}{5}u_{1}
\end{displaymath}
Solving these last two equations gives $u_{1}=\nunit{3.34}{\mm}$ and
$v_{1}=\nunit{-1.15}{\mm}$.  Thus the strain in truss $1$ is
$(\dfrac{\sqrt{3}}{2}3.34-\dfrac{1}{2}1.15) \times \tento{-3} =0.232\%$, in
truss $2$ is $-0.115\%$ and in truss $3$ is zero.

The tension in truss $1$ is $A\sigma = AEe = \nunit{5\times\tento{-3}}{\mtwo}
\nunit{\tento{7}}{\kPa} \times 0.232\times\tento{-2} = \nunit{116}{\kN}$ (tensile),
in truss $2$ is \nunit{-57.5}{\kN} (compressive) and in truss $3$ is zero. The
nodal reaction forces are shown in \Figref{fig:rf}.
    
\begin{figure}[htbp] \centering
  \input{figs/lin_elasticity/reaction.pstex}
  \caption{Reaction forces for the truss system of \Figref{fig:system}.}
  \label{fig:rf}
\end{figure}

%\begin{example}{Solving a truss system}
%  {Solving a truss system}

%  \todo{example???}

%  We solve the simple three truss system described above and shown in
%  \Figref{fig:system}. 

%  \label{xmp:Solving}
%\end{example}


%\begin{example}{Stresses in a bicycle frame modelled with truss elements}
%  {Stresses in a bicycle frame modelled with truss elements}

%  \todo{example???}

%  \begin{figure}[htbp] \centering
%    \input{figs/lin_elasticity/CMISSA.pstex}
%  \end{figure}
%  \label{xmp:stressesinabike}
%\end{example}
%\index{Truss elements|)}

\section{Beam Elements}

\index{Beam elements|(}
Simple beam theory ignores all but axial strain $e_{x}$  and stress 
$\sigma_{x}=Ee_{x}$  ($E=$ Young's modulus) along the beam (assumed here to be 
in the x-direction). The axial strain is given by $e_{x} = \dfrac{z}{R}$ ,
where $z$ is the lateral distance from the neutral axis in the plane of the 
bending and $R$ is the radius of curvature in that plane. The bending moment 
is given by $M= \gint{}{}{\sigma_{x} z}{A}$ , where $A$ is the beam crossectional 
area. Thus 
\begin{equation}
  \sigma_{x} =  Ee_{x} = E\dfrac{z}{R}
  \label{eqn:bem1}
\end{equation}
\begin{equation}
  M  = \gint{}{}{\sigma_{x} z}{A} = \frac{E}{R} \gint{}{}{z^{2}}{A} 
     = \dfrac{EI}{R}
  \label{eqn:bem2}
\end{equation}
where $I = \gint{}{}{z^{2}}{A}$ is the second moment of area of the beam
cross-section. Thus, $\dfrac{E}{R} = \dfrac{M}{I}$ and \eqnref{eqn:bem1} becomes
\begin{equation}
  \sigma_{x} = \dfrac{Mz}{I}
  \label{eqn:bem3}
\end{equation}
The slope of the beam is  $\dby{w}{x} = \theta$  and the rate of change of 
slope is the curvature 
\begin{equation}
  K=\dby{\theta}{x} = \dtwosqby{w}{x} = \dfrac{1}{R}
  \label{eqn:slope}
\end{equation}
Thus the bending moment is 
\begin{equation}
  M=EI \dtwosqby{w}{x} = EIw''
  \label{eqn:bmom}
\end{equation}
and a force balance gives the shear force
\begin{equation}
  V=-\dby{M}{x} = - \dby{ }{x}\pbrac{EIw''}
  \label{eqn:shf}
\end{equation}
and the normal force (per unit length of beam)
\begin{equation}
  p=\dby{V}{x} = -\dtwosqby{ }{x}\pbrac{EIw''}
  \label{eqn:normf}
\end{equation}
This last equation is the equilibrium equation for the beam, balancing the 
loading forces $p$ with the axial stresses associated with beam flexure
\begin{equation}
  -\dtwosqby{ }{x} \pbrac{EI \dtwosqby{w}{x}} = p
  \label{eqn:lasteq}
\end{equation}

The elastic strain energy stored in a bent beam is the sum of flexural strain
energy and shear strain energy, but this latter is ignored in the simple beam
theory considered here. Thus, the (flexural) strain energy is
\begin{align*}
  \text{SE} &= \dfrac{1}{2} \gint{x=0}{L}{\goneint{\sigma_{x}e_{x}}{A}}{x} =  
  \dfrac{1}{2} \gint{x=0}{L}{E\goneint{e_{x}^{2}}{A}}{x} \\
  &= \dfrac{1}{2} \gint{x=0}{L}{E \goneint{\pbrac{\dfrac{z}{R} 
  }^{2}}{A}}{x} =  \dfrac{1}{2} \gint{x=0}{L}{EI\pbrac{w''}^{2}}{x}
\end{align*}    
where $x$ is taken along the beam and $A$ is the cross-sectional area of the 
beam.

The external work associated with forces $p$ acting normal to the beam and
moving through a transverse displacement $w$ is $\gint{0}{L}{pw}{x}$.
The potential energy is therefore
\begin{equation}
  \text{PE}=\dfrac{1}{2} \gint{0}{L}{EI\pbrac{w''}^{2}}{x} - \gint{0}{L}{pw}{x}.
  \label{eqn:poten}
\end{equation}

The finite element approximation for the transverse displacement $w$ must be
able to represent the second derivative $w''$. A linear basis function has a
zero second derivative and therefore cannot represent the flexural strain. The
natural choice of basis function for beam deflection is in fact cubic Hermite
because the inter-element slope continuity of this basis ensures transmission
of bending moment as well as shear force across element boundaries.

The boundary conditions associated with the \nth{4} order equilibrium
\eqnref{eqn:lasteq} or the equations arising from minimum potential energy
\eqnref{eqn:poten} (which contain the square of $2^{\text{nd}}$ derivative terms) 
are more complex than the simple temperature or flux boundary conditions for the
(second order) heat equation. Three possible combinations of boundary
condition with their associated reactions are

\begin{tabular}{llll}
  && \emph{Boundary conditions} & \emph{Reactions}\\
  && \\
  (i) &  Simply supported &  zero displacement $w=0$ &  shear force $V$\\
  && zero moment $M = EIw'' = 0$ &  slope $\theta (= w')$\\
  (ii) &  Cantilever & zero displacement $w=0$ &  shear force $V$\\
  && zero slope  $\theta = w' = 0$ & moment $M$\\
  (iii)&  Free end & zero shear force $V=-\dby{ }{x}\pbrac{EIw''} = 0$ & 
  displacement $w$\\
  && zero moment $M = EI'' = 0$ & slope $\theta$
\end{tabular}

%\begin{example}{Stresses in a bicycle frame modelled with beam elements}
%  {Stresses in a bicycle frame modelled with beam elements.}

%  \todo{example???}

%  \begin{figure}[htbp] \centering
%    \input{figs/CMISSA.pstex}
%  \end{figure}

%  \label{xmp:Sbfbe}
%\end{example}
%\index{Beam elements|)}

\section{Plane Stress Elements}
\vspace{-5mm}
\index{Plane stress elements|(}
For two-dimensional problems, we define the displacement vector 
$\vect{u} = \begin{bmatrix} 
  u \\ 
  v 
\end{bmatrix}$, strain vector $\vect{e} = \begin{bmatrix}
  e_{x} \\
  e_{y} \\ 
  e_{xy}
\end{bmatrix}$ and stress vector $\vect{\sigma} = \begin{bmatrix}
  \sigma_{x} \\
  \sigma_{y} \\
  \sigma_{xy} 
\end{bmatrix}$. The stress-strain relation for two-dimensional plane stress:
\begin{gather}
  \begin{aligned}
    \sigma_{x} &= \dfrac{E}{1-\nu^{2}}\pbrac{e_{x}+\nu e_{y}} \\
    \sigma_{y} &= \dfrac{E}{1-\nu^{2}}\pbrac{e_{y}+\nu e_{x}} \\
    \sigma_{xy} &= \dfrac{E}{1+\nu}\pbrac{e_{xy}}
  \end{aligned}
  \label{eqn:planestress}
\end{gather}
can be written in matrix form
\begin{displaymath}
  \vect{\sigma}=\matr{E}\vect{e}
\end{displaymath}
where $\matr{E} = \dfrac{E}{1-\nu^{2}} \begin{bmatrix}
  1 & \nu & 0\\
  \nu & 1 & 0\\
  0 & 0 & 1-\nu
\end{bmatrix}$. 
The strain components are given in terms of displacement gradients by
\begin{equation}
  \begin{array}{rcl}
    e_{x} & = & \delby{u}{x} \\
    e_{y} & = & \delby{v}{y} \\
    e_{xy} & = & \dfrac{1}{2} \pbrac{\delby{u}{y} + \delby{v}{x}}
  \end{array}
  \label{eqn:stcom}
\end{equation}

The strain energy\index{Strain energy} is 
\begin{align*}
  \text{SE} &= \dfrac{1}{2} \goneint{\transpose{\vect{\sigma}}\vect{e}}{V} = \dfrac{1}{2}
  \goneint{\pbrac{e_{x}\sigma_{x} + e_{y}\sigma_{y} + e_{xy}\sigma_{xy}}}{V} \\ &=
  \dfrac{1}{2} \goneint{\transpose{\vect{e}}\matr{E}\vect{e}}{V} = \dfrac{1}{2}
  \goneint{\dfrac{E}{1-\nu^{2}} \sqbrac{e_{x}^{2} + e_{y}^{2} + 2\nu e_{x}e_{y} +
      \pbrac{1-\nu}e_{xy}^{2}}}{V}
\end{align*}

The potential energy\index{Potential energy} is
\begin{equation}
  \text{PE} = \text{SE} -\text{ external work }= \dfrac{1}{2}
  \goneint{\transpose{\vect{e}}\matr{E}\vect{e}}{V} -
  \goneint{\transpose{\vect{u}}\vect{l}}{A} 
  \label{eqn:pe}
\end{equation}
where $\vect{l}$ represents the external loads (forces) acting on the elastic
body.

Following the steps outlined in \Secref{sec:Intro} we approximate the
displacement field $\vect{u}$ with a finite element basis $u=\lbfnsymb{n} u_{n}$,
$v=\lbfnsymb{n} v_{n}$ and calculate the strains
\begin{gather}
  \begin{aligned}
    e_{x} &= \delby{u}{x} = \delby{\lbfnsymb{n}}{x} u_{n} \\
    e_{y} &= \delby{v}{y} = \delby{\lbfnsymb{n}}{y} v_{n} \\
    e_{xy} &= \dfrac{1}{2}\pbrac{\delby{u}{y}+ \delby{v}{x}} = 
    \dfrac{1}{2} \pbrac{\delby{\lbfnsymb{n}}{y}u_{n} + 
      \delby{\lbfnsymb{n}}{x}v_{n}}
  \end{aligned}
  \label{eqn:strains}
\end{gather}
or
\begin{equation}
  \vect{e} = \begin{bmatrix}
    e_{x} \\
    e_{y} \\
    e_{xy}
    \end{bmatrix} = \begin{bmatrix}
      \delby{\lbfnsymb{n}}{x} & 0 \\
      0 & \delby{\lbfnsymb{n}}{y} \\
      \dfrac{1}{2}\delby{\lbfnsymb{n}}{y} & \dfrac{1}{2}\delby{\lbfnsymb{n}}{x}
  \end{bmatrix} \begin{bmatrix}
    u_{n} \\
    v_{n} 
  \end{bmatrix} = \matr{B}\vect{u}
  \label{eqn:orstr}
\end{equation}
From \eqnref{eqn:pe} the potential energy is therefore
\begin{align*}
  \text{PE} &= \dfrac{1}{2}
  \goneint{\transpose{\pbrac{\matr{B}\vect{u}}}\matr{E}\pbrac{\matr{B}\vect{u}}}{V} -
  \goneint{\transpose{\vect{u}}\vect{l}}{A} \\ 
  &= \dfrac{1}{2}\transpose{\vect{u}} \left[
  \goneint{\transpose{\matr{B}}\matr{E}\matr{B}}{V} \right] \vect{u} - 
  \goneint{\transpose{\vect{u}}\vect{l}}{A} \\
  &= \dfrac{1}{2} \transpose{\vect{u}}\matr{K}\vect{u}-
  \goneint{\transpose{\vect{u}}\vect{l}}{A}
\end{align*}
where $\matr{K} = \goneint{\transpose{\matr{B}}\matr{E}\matr{B}}{V}$ is the
element stiffness matrix.

We next minimize the potential energy with respect to the nodal parameters
$u_{n}$ and $v_{n}$ giving
\begin{equation}
  \matr{K}\vect{u} = \vect{f}
  \label{eqn:pennodpam}
\end{equation}
\index{Plane stress elements|)} 
where $\vect{f} = \goneint{\vect{l}}{A}$ is a vector of nodal forces.

\subsection{Notes on calculating nodal loads}
\label{sec:noteoncalc}

If a known stress acts normal to a given surface (\eg a surface pressure), it
may be applied by calculating equivalent nodal forces. For example, consider a
uniform load $\nunit{p}{\kNpm}$ applied to the edge of the plane stress
element in \Figref{fig:uniformbound}a.

The nodal load vector $\vect{f}$ in \eqnref{eqn:pennodpam} has components
\begin{equation}
  f_{n} = \goneint{p \lbfnsymb{n}}{x} = pL \gint{0}{1}{\lbfnsymb{n}}{\xi}
  \label{eqn:nlv}
\end{equation}  
where $\xi$ is the normalized element coordinate along the side of length $L$
loaded by the constant stress \nunit{p}{\kNpm}. If the element side has a
linear basis, \eqnref{eqn:nlv} gives
\begin{align*}
  f_{1} &= pL\gint{0}{1}{\lbfnsymb{1}}{\xi} = pL\gint{0}{1}{\pbrac{1-\xi}}{\xi} =
  \dfrac{1}{2} pL \\
  f_{2} &= pL\gint{0}{1}{\lbfnsymb{2}}{\xi} = pL\gint{0}{1}{\xi}{\xi} = \dfrac{1}{2} pL
\end{align*}
as shown in \Figref{fig:uniformbound}b. If the element side has a
quadratic basis, \eqnref{eqn:nlv} gives 
\begin{align*}
  f_{1} &= pL\gint{0}{1}{\lbfnsymb{1}}{\xi} =
  pL\gint{0}{1}{2\pbrac{\dfrac{1}{2}-\xi}\pbrac{1-\xi}}{\xi} = \frac{1}{6} pL \\
  f_{2} &= pL\gint{0}{1}{\lbfnsymb{2}}{\xi} = pL \gint{0}{1}{4\xi\pbrac{1-\xi}}{\xi}  
  = \frac{2}{3} pL \\
  f_{3} &= pL\gint{0}{1}{\lbfnsymb{3}}{\xi} = pL \gint{0}{1}{2\xi\pbrac{\xi -
      \dfrac{1}{2}}}{\xi} = \frac{1}{6} pL
\end{align*}
as shown in \Figref{fig:uniformbound}c. A node common to two elements will
receive contributions from both elements, as shown in
\Figref{fig:uniformbound}d.

\begin{figure}[htbp] \centering
  \input{figs/lin_elasticity/uniformbound.pstex}
  \caption{A uniform boundary stress applied to the element side in (a) is 
   equivalent to nodal loads of $\frac{1}{2} pL$ and $\frac{1}{2} pL$ for the 
   linear basis used in (b) and to  $\frac{1}{6}pL$,  $\frac{2}{3}pL$ and  
   $\frac{1}{6}pL$ for the quadratic basis used in (c). Two adjacent quadratic
   elements both contribute to a common node in (d), where the element 
   length is now $\frac{L}{2}$.}
  \label{fig:uniformbound}
\end{figure}


\section{Three-Dimensional Elasticity}
\label{sec:3Delast}

Consider a surface $\Gamma$ enclosing a volume $\Omega$ of material of mass
density $\rho$.  Conservation of linear momentum over the domain $\Omega$
results in the governing stress equilibrium equations
\begin{equation}
  \sigma_{ij,j} + b_{i} =  0 \qquad i,j=1,2,3
  \label{eqn:equilib}
\end{equation}
where $\sigma_{ij}$ are the components of the stress tensor ($\sigma_{ij}$ is
the component of the traction or stress vector in the $\nth{i}$ direction
which is acting on the face of a rectangle whose normal is in the $\nth{j}$
direction), and $b_{i}$ is the body force/unit volume (\eg $\vect{b} =
\rho\vect{g}$). Note that the notation $\sigma_{ij,j} =
\delby{\sigma_{ij}}{x_{j}}$ has been introduced to represent the derivative.

Recall that the components of the linear (or small) strain tensor are
\begin{equation}
  e_{ij} =  \dfrac{1}{2}\pbrac{u_{i,j} + u_{j,i}} \qquad i,j=1,2,3
  \label{eqn:sst}
\end{equation}
where $\vect{u}$ is the displacement vector (\ie $\vect{u}$ is the difference
between the final and initial positions of a material point in
question). Note: we are assuming here that the displacement gradients are
small compared to unity, which is appropriate for many materials in solid
mechanics.  However, for soft materials, such as rubber or biological tissue,
then we need to use the exact finite strain tensor.

The object of solving an elasticity problem is to find the distributions of
stress and displacement in an elastic body, subject to a known set of body
forces and prescribed stresses or displacements at the boundaries.  In the
general three-dimensional case, this means finding $6$ stress components
$\sigma_{ij}$ ($=\sigma_{ji}$ which arises from the conservation of angular
momentum) and 3 displacements $u_{i}$ each as a function of position in the
body. Currently we have $15$ unknowns ($6$ stresses, $6$ strains and $3$
displacements), but only $9$ equations ($3$ equilibrium equations and $6$
strain-displacement relations).

To progress, we require an equation of state, \ie a stress-strain relation or
constitutive law.  For a linear elastic material we may propose that the
components of stress $\sigma_{ij}$ depend linearly on $e_{ij}$. \ie
\begin{displaymath}
  \sigma_{ij} = c_{ijkl}e_{kl}
\end{displaymath}  
where $c_{ijkl}$ are the $81$ components of a $\nth{4}$ order tensor, although
symmetry of the strain and stress tensors reduces the number of independent
components to $21$.

If the material is assumed to be isotropic (\ie the material response is
independent of orientation of the material element), then we end up with the
generalized Hooke's Law.
\begin{equation}
  \sigma_{ij} = \lambda e_{kk} \delta_{ij} + 2 \mu e_{ij}
  \label{eqn:genHook}
\end{equation}
or inversely
\begin{displaymath}
  e_{ij} = \dfrac{1}{2 \mu} \sigma_{ij} - \dfrac{\lambda}{2 \mu \pbrac{3 \lambda 
    + 2 \mu}} \sigma_{kk} \delta_{ij}
\end{displaymath}
where $\lambda$, $\mu$ are Lam\'{e}s constants.

Note: $\lambda$, $\mu$ are related to Young's modules $E$ and Poisson's 
ratio $\nu$ by 
\begin{displaymath}
  E  =  \dfrac{\mu \pbrac{3 \lambda + 2 \mu}}{\lambda + \mu}
\end{displaymath}
\begin{displaymath}
  \nu = \dfrac{\lambda}{2\pbrac{\lambda + \mu}}
\end{displaymath}   

Providing that the displacements are continuous functions of position, then
\eqnref{eqn:equilib}, \eqnref{eqn:sst} and \eqnref{eqn:genHook} are sufficient
to determine the $15$ unknown quantities.  This can often work with some
smaller grouping or simplification of these equations, \eg if all boundary
conditions are expressed in terms of displacements, substituting
\eqnref{eqn:sst} into \eqnref{eqn:genHook} then into \eqnref{eqn:equilib}
yields Navier's equation of motion.
\begin{displaymath}
  \mu u_{i,kk} + \pbrac{\lambda + \mu} u_{k,ki} + b_{i} = 0 \qquad i,k=1,2,3
\end{displaymath}
These $3$ equations can be solved for the unknown displacements. Then
\eqnref{eqn:sst} can be used to determine the strains and \eqnref{eqn:genHook}
to calculate the stresses.


\subsection{Weighted Residual Integral Equation}
Using weighted residuals as before we can write
\begin{equation}
  \goneint{\pbrac{\sigma_{ij,j} + b_{i}} u_{i}^{*}}{\Omega} = 0
  \label{eqn:integral}
\end{equation}
where $\vect{u}^{*} = \pbrac{u_{i}^{*}}$ is a (vector) weighting field.  The
$u_{i}^{*}$ are usually interpreted as a consistent set of virtual
displacements (hence we use the notation $u$ instead of $w$).

By the chain-rule
\begin{displaymath}
  \pbrac{\sigma_{ij}u_{i}^{*}}_{,j} = \sigma_{ij,j}u_{i}^{*} +
  \sigma_{ij}u_{i,j}^{*}
\end{displaymath}

Therefore, the first term in the integrand of \eqnref{eqn:integral} can be
re-written
\begin{align}
  \goneint{\sigma_{ij,j} u_{i}^{*}}{\Omega}
  &= \goneint{(\sigma_{ij} u_{i}^{*})_{,j}}{\Omega} - 
  \goneint{\sigma_{ij}u_{i,j}^{*}}{\Omega} \nonumber \\
%%
  &= \goneint{\diverg{\pbrac{\sigma_{ij} u_{i}^{*} }}}{\Omega}
    - \goneint{\sigma_{ij} u_{i,j}^{*}}{\Omega} \nonumber\\
%%
  &= \gint{\del \Omega}{}{\sigma_{ij} u_{i}^{*}n_{j}}{\Gamma}
  - \goneint{\sigma_{ij} u_{i,j}^{*}}{\Omega}
  \label{eqn:divtheory}
\end{align}
where the domain integral involving ``$\diverg{} = \delby{}{x_{j}}$'' has been
transformed into a surface integral using the divergence theorem
\begin{displaymath}
  \goneint{\diverg{\vect{g}}}{\Omega} = \gint{\del
  \Omega}{}{\dotprod{\vect{g}}{\vect{n}}}{\Gamma} \qquad \text{or} \qquad
  \goneint{g_{j,j}}{\Omega} = \gint{\del \Omega}{}{g_{j} n_{j}}{\Gamma}
\end{displaymath}
where $\vect{n} = n_{j}\vect{i}_{j}$ is the outward normal vector to the
surface $\Gamma$.

Thus, combining \eqnref{eqn:integral} and \eqnref{eqn:divtheory} we have
\begin{align}
  \goneint{\sigma_{ij} u_{i,j}^{*}}{\Omega} &=
  \goneint{b_{i} u_{i}^{*}}{\Omega} + \gint{\del \Omega}{}
  {\sigma_{ij} n_{j}u_{i}^{*}}{\Gamma} \nonumber \\ &= \goneint
  {b_{i} u_{i}^{*}}{\Omega} + \gint{\del \Omega}{}{t_{i} u_{i}^{*}}{\Gamma}
  \label{eqn:virtws}
\end{align}
where $t_{i}$ are the components of the internal stress vector (\vect{t}) and
are related to the components of the stress tensor ($\sigma_{ij}$) by Cauchy's
formula
\begin{equation}
\vect{t} = \sigma_{ij}n_{j}\vect{i}_{i}
\label{eqn:cauchyseqn}
\end{equation}

To arrive at this point, we have used weighted residuals to tie in with
\chapref{cha:steadystate}, however \eqnref{eqn:virtws} is more usually derived
using the principle of virtual work (below). Note that the weighted integral
\eqnref{eqn:virtws} is independent of the constitutive law of the material.


\subsection{The Principle of Virtual Work}

The governing equations for elastostatics can also be derived from a
physically appealing argument. Let $\vect{s}$ be the external traction vector
(\ie force per unit surface area).  For equilibrium, the work done by the
external surface forces $\vect{s} = s_{i}\vect{i}_{i}$, in moving through a
virtual displacement $\vect{u}^{*} = u_{i}^{*}\vect{i}_{i}$ is equal to the
work done by the stress vector $\vect{t} = t_{i}\vect{i}_{i}$ in moving
through a compatible set of virtual displacements $\vect{u}^{*}$. In
mathematical terms, the principle of virtual work can be written
\begin{equation}
  \gint{\del \Omega}{}{s_{i} u_{i}^{*}}{\Gamma} = \gint{\del \Omega}{}{t_{i} u_{i}^{*}}{\Gamma} 
  = \gint{\del \Omega}{}{\sigma_{ij}n_{j} u_{i}^{*}}{\Gamma}
  \label{eqn:virtwork1}
\end{equation}
using Cauchy's formula (\eqnref{eqn:cauchyseqn}).

The Green-Gauss theorem (\eqnref{eqn:nabla_eq}) is now used to replace the
right hand surface integral in \eqnref{eqn:virtwork1} by a volume integral,
giving
\begin{equation}
  \gint{\del \Omega}{}{s_{i} u_{i}^{*}}{\Gamma} = \goneint{\pbrac{
  \sigma_{ij,j} u_{i}^{*} + \sigma_{ij} u_{i,j}^{*} }}{\Omega}
\end{equation}

Substituting the equilibrium relation (\eqnref{eqn:equilib}) into the first
integrand on the right hand side, yields the \emph{virtual work} equation
\begin{equation}
  \goneint{\sigma_{ij} u_{i,j}^{*}}{\Omega} = \goneint{b_{i}
  u_{i}^{*}}{\Omega} + \gint{\del \Omega}{}{s_{i} u_{i}^{*}}{\Gamma}
  \label{eqn:virtwork2}
\end{equation}
where the internal work done due to the stress field is equated to the work
due to internal body forces and external surface forces.  Note that
\eqnref{eqn:virtwork2} is equivalent to \eqnref{eqn:virtws} via
\eqnref{eqn:virtwork1}. In practice, \eqnref{eqn:virtwork2} is in a more
useful form than \eqnref{eqn:virtws}, because the right hand side integrals
can be expressed in terms of the known body forces and the applied boundary
conditions (surface traction forces or stresses).


\subsection{The Finite Element Approximation}

Let $\Omega = \bigcup\thinspace\Omega_{e}$ and interpolate the virtual
displacements $u_{i}^{*}$ from their nodal values. \ie
\begin{gather}
  \begin{aligned}
    u_{i}^{*} &= \lbfnsymb{m} \dot (u_{i}^{m})^{*} \\
    \text{so }\quad u_{i,j}^{*} &= \delby{\lbfnsymb{m}}{x_{j}} \dot (u_{i}^{m})^{*} \\
    &= \lbfnsymb{m,k} \delby{\xi_{k}}{x_{j}} \dot (u_{i}^{m})^{*}    
  \end{aligned}
  \label{eqn:virtdisplapprox}
\end{gather}
where $(u_{i}^{m})^{*} = (U_{i}^{\triangle(m,e)})^{*}$, $\triangle(m,e)$ is
the global node number of local node $m$ on element $e$, and the shorthand
$\lbfnsymb{m,k}=\delby{\lbfnsymb{m}}{\xi_{k}}$ has been introduced.

Substituting this into \eqnref{eqn:virtwork2} gives
\begin{equation*}
  \sum_{e} \pbrac{
    \gint{\thickspace\Omega_{e}}{}{\sigma_{ij}\lbfnsymb{m,k}\delby{\xi_{k}}{x_{j}}}
    {\Omega} } \pbrac{ U_{i}^{\triangle(m,e)} }^{*} = \sum_{e} \pbrac{
    \gint{\Omega_{e}}{}{b_{i}\lbfnsymb{m}}{\Omega} + \gint{\del
    \Omega_{e}}{}{s_{i}\lbfnsymb{m}}{\Gamma} } \pbrac{U_{i}^{\triangle(m,e)}
    }^{*}
\end{equation*}
and since the virtual displacements are arbitrary we get
\begin{equation}
  \sum_{e}\gint{\Omega_{e}}{}{\sigma_{ij}\lbfnsymb{m,k}\delby{\xi_{k}}{x_{j}}}{\Omega}
    = \sum_{e} \pbrac{\gint{\Omega_{e}}{}{b_{i}\lbfnsymb{m}}{\Omega} 
    + \gint{\del \Omega_{e}}{}{s_{i}\lbfnsymb{m}}{\Gamma}}
  \label{eqn:intequilib}
\end{equation}

The next step is to express the stress components $\sigma_{ij}$ in terms of
the virtual displacements and their finite element approximation by
substituting \eqnref{eqn:virtdisplapprox} into \eqnref{eqn:sst} (the
strain-displacement relation) and in turn into \eqnref{eqn:genHook} (the
generalized Hooke's law).

We first introduce the finite element approximation for the displacement field
$u_{j}=\lbfnsymb{n}u_{j}^{n}$ which gives
\begin{equation}
  e_{ij} = \frac{1}{2}\pbrac{\hdelby{\lbfnsymb{n}u_{i}^{n}}{x_{j}} +
  \hdelby{\lbfnsymb{n}u_{j}^{n}}{x_{i}}} =
  \frac{1}{2}\pbrac{\delby{\lbfnsymb{n}}{\xi_{l}}\delby{\xi_{l}}{x_{j}}u_{i}^{n}
  + \delby{\lbfnsymb{n}}{\xi_{l}}\delby{\xi_{l}}{x_{i}}u_{j}^{n}}
\end{equation}
and
\begin{equation*}
  e_{kk} = u_{k,k} = \delby{\lbfnsymb{n}}{\xi_{l}}\delby{\xi_{l}}{x_{k}}u_{k}^{n}
\end{equation*}
Thus
\begin{equation*}
  \sigma_{ij} = \lambda \delta_{ij}
  \delby{\lbfnsymb{n}}{\xi_{l}}\delby{\xi_{l}}{x_{k}}u_{k}^{n} + 2\mu \pbrac{
  \frac{1}{2}\delby{\lbfnsymb{n}}{\xi_{l}}\delby{\xi_{l}}{x_{j}}u_{i}^{n} +
  \frac{1}{2}\delby{\lbfnsymb{n}}{\xi_{l}}\delby{\xi_{l}}{x_{i}}u_{j}^{n}}
\end{equation*}
which, due to symmetry of the stress tensor, simplifies to
\begin{align}
  \sigma_{ij} &= \lambda \delta_{ij}
  \delby{\lbfnsymb{n}}{\xi_{l}}\delby{\xi_{l}}{x_{k}}u_{k}^{n} + 2\mu 
  \delby{\lbfnsymb{n}}{\xi_{l}}\delby{\xi_{l}}{x_{i}}u_{j}^{n} \nonumber \\
              &= \pbrac{\lambda \delta_{i(j)}
  \lbfnsymb{n,l}\delby{\xi_{l}}{x_{j}} + 2\mu 
  \lbfnsymb{n,l}\delby{\xi_{l}}{x_{i}}}u_{j}^{n}
  \label{eqn:stresscmpts}
\end{align}
where the summation index $k$ has been replaced with $j$, but the parenthesis
in $\delta_{i(j)}$ implies that there is no sum with respect to that
particular index.

Substituting this expression into \eqnref{eqn:intequilib} and simplifying, we
get for each element
\begin{equation}
    u_{j}^{n} \gint{\Omega_{e}}{}{\pbrac{ \lambda
       \lbfnsymb{n,l}\delby{\xi_{l}}{x_{j}}
       \lbfnsymb{m,k}\delby{\xi_{k}}{x_{i}} + 2\mu
       \lbfnsymb{n,l}\delby{\xi_{l}}{x_{i}}
       \lbfnsymb{m,k}\delby{\xi_{k}}{x_{j}}}}{\Omega} = f_{im}
  \label{eqn:linelasfem}
\end{equation}
where $f_{im}$ denotes the right hand side terms in
\eqnref{eqn:intequilib}. (Note that there has been some careful manipulation
of summation indices with the substitution of \eqnref{eqn:stresscmpts} to
arrive at \eqnref{eqn:linelasfem}.)

So for each element
\begin{equation*}
  E_{imjn} u_{j}^{n} = f_{im}
\end{equation*}
where
\begin{gather}
  \begin{aligned}
    E_{imjn} &= \iiint\limits_{0}^{1} \pbrac{ \lambda
       \delby{\xi_{l}}{x_{j}}\delby{\xi_{k}}{x_{i}} + 2\mu
       \delby{\xi_{l}}{x_{i}}\delby{\xi_{k}}{x_{j}}} 
       \lbfnsymb{n,l} \lbfnsymb{m,k} J(\xione,\xitwo,\xithree) d\xione
       d\xitwo d\xithree \\ 
    f_{im} &= \iiint\limits_{0}^{1} b_{i}\lbfnsymb{m}
       J(\xione,\xitwo,\xithree) d\xione d\xitwo d\xithree +
       \iint\limits_{0}^{1} s_{i}\lbfnsymb{m} 
       J_{2D}(\xione,\xitwo) d\xione d\xitwo
  \end{aligned}
\end{gather}
where the Jacobians $J(\xione,\xitwo,\xithree)$ and $J_{2D}(\xione,\xitwo)$
have been used to transform volume and surface integrals so that they can be
can be calculated using $\xi$-coordinates. (Note: without loss of generality,
the above definition of $f_{im}$ assumes that $(\xione,\xitwo)$ are defined to
lie in the surface $\Gamma$.)

So in summary, the finite element approximation leads to element stiffness
matrix components that can be calculated from the known material parameters,
the chosen interpolation functions, and the geometry of the material (note
that the element stiffness components are independent of the unknown
displacement parameters).  Element stiffness components are then assembled
into the global stiffness matrix in the usual manner (as described
previously).  Note that this is implicitly a Galerkin formulation, since the
unknown displacement fields are interpolated using the same basis functions as
those used to weight the integral equations.

%\begin{example}{Stresses in a plate with a hole}
%  {Stresses in a plate with a hole}
  
%  \todo{example???}

%  A common problem in structural mechanics is to find the stress concentration
%  produced by a hole in an otherwise uniformly loaded structural component.
%  Consider the plate below loaded by horizontal forces of \nunit{100}{\kNpm}.
%  The plate is \nunit{1}{\m} thick and made of steel (Young's modulus
%  \nunit{100}{\GPa}, Poisson's ratio $0.3$).

%  \begin{figure}[htbp] \centering
%    \input{figs/lin_elasticity/CMISSB.pstex}
%  \end{figure}

%  \label{xmp:stressesinplate}
%\end{example}


\section{Linear Elasticity with Boundary Elements}
\label{sec:Lewbe-4.8}

\Eqnref{eqn:virtws} is the starting point for the general finite element
formulation (\Secref{sec:3Delast}). In the above derivation, we have 
essentially used the Green-Gauss theorem once to move from \eqnref{eqn:integral} 
to \eqnref{eqn:virtws} (as was done for the derivation of the FEM equation for
Laplace's equation). To continue, we firstly note that
\begin{align*}
  \sigma_{ij}e_{ij}^{*} &= \dfrac{1}{2}\sigma_{ij}u_{i,j}^{*} +
  \dfrac{1}{2}\sigma_{ij}u_{j,i}^{*}\\ 
  &= \dfrac{1}{2}\sigma_{ij}u_{i,j}^{*} + \dfrac{1}{2}\sigma_{ji}u_{j,i} ^{*} \\ 
  &= \dfrac{1}{2}\sigma_{ij}u_{i,j}^{*} + \dfrac{1}{2}\sigma_{ij}u_{i,j} ^{*} \\ 
  &= \sigma_{ij} u_{i,j}^{*}
\end{align*}
where $e_{ij}^{*}$ are the virtual strains corresponding to the virtual
displacements.

Using the constitutive law for linearly elastic materials
(\eqnref{eqn:genHook}) we have
\begin{align*}
  \goneint{\sigma_{ij}u_{i,j}^{*}}{\Omega} &=
  \goneint{\sigma_{ij} e_{ij}^{*}}{\Omega} \\ 
  &= \lambda\goneint{e_{kk} e_{ij}^{*} \delta_{ij}}{\Omega} + 
  2\mu\goneint{e_{ij} e_{ij}^{*}}{\Omega} \\ 
  &= \lambda\goneint{e_{kk} e_{kk}^{*}}{\Omega} + 
  2\mu \goneint{e_{ij} e_{ij}^{*}}{\Omega} \\ 
  &= \goneint{e_{ij} \sigma_{ij}^{*}}{\Omega}
\end{align*}
due to symmetry.

Thus from the virtual work statement, \eqnref{eqn:virtws} and the above
symmetry we have
\begin{equation}
  \goneint{b_{i}u_{i}^{*}}{\Omega} + \gint{\del \Omega}{} 
  {t_{i}u_{i}^{*}}{\Gamma} = \goneint{b_{i}^{*}u_{i}}{\Omega} + 
  \gint{\del \Omega}{}{t_{i}^{*} u_{i}}{\Gamma}
  \label{eqn:Bsrwt}
\end{equation}
This is known as Betti's second reciprical work theorem or the Maxwell-Betti
reciprocity relationship between two different elastic problems (the starred
and unstarred variables) established on the same domain.

Note that $b_{i}^{*} = -\sigma_{ij,j} ^{*}$  (\ie $\sigma_{ij,j} ^{*} +
b_{i}^{*} =  0$). Therefore \eqnref{eqn:Bsrwt} can be written as
\begin{equation}
  \goneint{\pbrac{\sigma_{ij,j} ^{*}} u_{i}}{\Omega} + 
  \goneint{b_{i} u_{i}^{*}}{\Omega} = 
  \gint{\del \Omega}{}{t_{i}^{*} u_{i}}{\Gamma} - 
  \gint{\del \Omega}{}{t_{i} u_{i}^{*}}{\Gamma}
  \label{eqn:eqnD}
\end{equation}
($\sigma_{ij} ^{*},e_{ij}^{*},t_{i}^{*}$ represents the equilibrium state
corresponding to the virtual displacements $u_{i}^{*}$).

Note: What we have essentially done is use integration of parts to get
\eqnref{eqn:virtws}, then use it again to get \eqnref{eqn:Bsrwt} above (after
noting the reciprocity between $\sigma_{ij}$ and $e_{ij}$).

Since the body forces, $b_{i}$, are known functions, the second domain
integral on the left hand side of \eqnref{eqn:eqnD} does not introduce any
unknowns into the problem (more about this later). The first domain integral
contains unknown displacements in $\Omega$ and it is this integral we wish to
remove.
 
We choose the virtual displacements such that
\begin{equation}
  \sigma_{ij,j}^{*} + e_{i} \delta = 0 
  \label{eqn:FundSol}
\end{equation}
(or equivalently $- b_{i}^{*} + e_{i} \delta = 0$), where $e_{i}$ is the \nth{i}
component of a unit vector in the \nth{i} direction and $e_{i}\delta =
e_{i}\fnof{\delta}{\vect{x} - P}$. We can interpret this as the body force
components which correspond to a positive unit point load applied at a point
$P \in \Omega$ in each of the three orthogonal directions.

Therefore

\begin{displaymath}
  \gint{\Omega}{}{\sigma_{ij,j}^{*} u_{i}}{\Omega} = -
  \gint{\Omega}{}{\fnof{\delta}{\vect{x} - P} e_{i} u_{i}}{\Omega} = - u_{i}(P)e_{i}
\end{displaymath}
\ie the volume integral is replaced with a point value (as for Laplace's 
equation).

Therefore, \eqnref{eqn:eqnD} becomes
\begin{equation}
  \fnof{u_{i}}{P}e_{i}  = \gint{\del \Omega}{}{t_{j} u_{j}^{*}}{\Gamma}  -  
  \gint{\del \Omega}{}{t_{j}^{*} u_{j}}{\Gamma} + 
  \goneint{b_{j} u_{j}^{*}}{\Omega} \qquad P \in \Omega
  \label{eqn:eqnE}
\end{equation}

If each point load is taken to be independent then $u_{j}^{*}$ and $t_{j}^{*}$
can be written as
\begin{align}
  u_{j}^{*} &= \fnof{u_{ij}^{*}}{P,x} e_{i} \\
  t_{j}^{*} &= \fnof{t_{ij}^{*}}{P,x} e_{i}
\end{align}
where $\fnof{u_{ij}^{*}}{P,x}$ and $\fnof{t_{ij}^{*}}{P,x}$ represent the
displacements and tractions in the $\nth{j}$ direction at $x$ corresponding to
a unit point force acting in the $\nth{i}$ direction ($e_{i}$) applied at $P$.
Substituting these into \eqnref{eqn:eqnE} (and equating components in each
$e_{i}$ direction) yields
\begin{multline}
  \fnof{u_{i}}{P} = \gint{\del \Omega}{}{\fnof{u_{ij}^{*}}{P,x}
    \fnof{t_{j}}{x}}{\fnof{\Gamma}{x}} - \gint{\del
    \Omega}{}{\fnof{t_{ij}^{*}}{P,x} \fnof{u_{j}}{x}}{\fnof{\Gamma}{x}} \\ 
  + \gint{\Omega}{}{\fnof{u_{ij}^{*}}{P,x}
    \fnof{b_{j}}{x}}{\fnof{\Omega}{x}}
 \label{eqn:eqnF}
\end{multline}

where $P \in \Omega$ (see later for $P \in \del\Omega$).

This is known as Somigliana's \footnote{Somigliana was an Italian
  Mathematician who published this result around 1894-1902.} identity for
displacement.

\section{Fundamental Solutions}

Recall from \eqnref{eqn:FundSol} that $\sigma_{ij}^{*}$ satisfied
\begin{equation}
  \sigma_{ij,j}^{*} + \fnof{\delta}{\vect{x} - P}e_{i}  = 0 
\end{equation} 
or equivalently
\begin{displaymath}
  b_{i}^{*} = e_{i} \fnof{\delta}{\vect{x} - P}
\end{displaymath}
Navier's equation for the displacements $u_{i}^{*}$ is
\begin{displaymath}
  G\;u_{i,kk}^{*}  +  \dfrac{G}{1-2\nu}u_{k,ki}^{*}  + b_{i}^{*} = 0 
\end{displaymath}
where $G$ = shear Modulus.

Thus $u_{i}^{*}$ satisfy
\begin{equation}
  G\;u_{i,kk}^{*}  +  \dfrac{G}{1-2\nu}u_{k,ki}^{*} + \fnof{\delta}{\vect{x} -
    P}e_{i} = 0 
  \label{eqn:satisfy}
\end{equation}

The solutions to the above equation in either two or three dimensions are
known as Kelvin~\footnote{Lord Kelvin (1824-1907) Scottish physicist who made
  great contributions to the science of thermodynamics}'s fundamental
solutions\index{Fundamental solution!Kelvin} and are given by
\begin{equation}
  \fnof{u_{ij}^{*}}{P,\vect{x}} = \dfrac{1}{16\pi\pbrac{1 - \nu}Gr}\bbrac{\pbrac{3
    -4\nu} \delta_{ij} + r_{,i}r_{,j}}
  \label{eqn:Kelvin}
\end{equation}
for three-dimensions and for two-dimensional plane strain problems,
\begin{equation}
  \fnof{u_{ij}^{*}}{P,\vect{x}} = \dfrac{-1}{8\pi\pbrac{1 -
      \nu}G}\bbrac{\pbrac{3 -4\nu} \delta_{ij} \log r - r_{,i}r_{,j}}
  \label{eqn:threeD}
\end{equation}
and 
\begin{equation}
  \fnof{t_{ij}^{*}}{P,\vect{x}} = \dfrac{-1}{4 \alpha \pi\pbrac{1 - \nu}r^{\alpha}}
  \bbrac{\pbrac{\pbrac{1-2\nu} \delta_{ij} + \beta r_{,i}r_{,j}} 
      \delby{r}{n} - \pbrac{1-2\nu}\pbrac{r_{,i}n_{j} -r_{,j}n_{i}}} 
  \label{eqn:twoD}
\end{equation}
where $\alpha = 1,2; \beta = 2,3$ for two-dimensional plane strain and
three-dimensional problems respectively.

Here $r\equiv \fnof{r}{P,\vect{x}}$, the distance between load point ($P$) and
field point ($\vect{x}$), 
$r_{i}  = \fnof{x_{i}}{\vect{x}}-\fnof{x_{i}}{P}$ and
$r_{,i} = \delby{r}{\fnof{x_{i}}{\vect{x}}}= \dfrac{r_{i}}{r}$.

In addition the strains at an point $\vect{x}$ due to a unit point load applied
at $P$ in the $\nth{i}$ direction are given by
\begin{displaymath}
  \fnof{e_{jki}^{*}}{P,\vect{x}} = \dfrac{-1}{8 \alpha \pi \pbrac{1-\nu}
    Gr^{\alpha}} \bbrac{\pbrac{1-2\nu}\pbrac{r_{,k} \delta_{ij} +r_{,j} \delta
    _{ik}} - r_{,i} \delta_{jk} + \beta r_{,i} r_{,j}r_{,k}}
\end{displaymath}
and the stresses are given by
\begin{displaymath}
  \fnof{\sigma_{ijk}^{*}}{P,\vect{x}} = \dfrac{-1}{4 \alpha \pi \pbrac{1-\nu}
    r^{\alpha}} \bbrac{\pbrac{1-2\nu}\pbrac{r_{,k} \delta_{ij} +r_{,j} \delta
      _{ki} - r_{,i} \delta_{jk}} + \beta r_{,i} r_{,j}r_{,k}}
\end{displaymath}
where $\alpha$ and $\beta$ are defined above.

The plane strain expressions are valid for plane stress if $\nu$ is replaced
by $\overline{\nu} = \dfrac{\nu}{1 + \nu}$ (This is a mathematical equivalence
of plane stress and plane strain - there are obviously physical differences.
What the mathematical equivalence allows us to do is to use one program to
solve both types of problems - all we have to do is modify the values of the
elastic constants).

Note that in three dimensions
\begin{displaymath}
  u_{ij}^{*} = \orderof{\frac{1}{r}}\hspace{0.5in} t_{ij}^{*} = 
  \orderof{\frac{1}{r^{2}}}
\end{displaymath}
 and for two dimensions
\begin{displaymath}
  u_{ij}^{*} = \orderof{\log r}\hspace{0.5in}  t_{ij}^{*} = \orderof{\frac{1}{r}}.
\end{displaymath}


Somigliana's identity (\eqnref{eqn:eqnF}) is a continuous representation of
displacements at any point $P \in \Omega$.  Consequently, one can find the
stress at any $P \in \Omega$ firstly by combining derivatives of
\bref{eqn:eqnF} to produce the strains and then substituting into Hooke's law.
Details can be found in \citeasnoun{brebbia:1984} pp 190--191, 255--258.

This yields
\begin{align*}
  \fnof{\sigma_{ij}}{P} &= \gint{\Gamma}{}{\fnof{u_{ijk}^{*}}{P,\vect{x}}
    \fnof{t_{k}}{\vect{x}}}{\fnof{\Gamma}{\vect{x}}} -
    \gint{\Gamma}{}{\fnof{t_{ijk}^{*}}{P,\vect{x}}
    \fnof{u_{k}}{\vect{x}}}{\fnof{\Gamma}{\vect{x}}} \\ 
  &+ \gint{\Omega}{}{\fnof{u_{ijk}^{*}}{P,\vect{x}}
    \fnof{b_{k}}{\vect{x}}}{\fnof{\Omega}{\vect{x}}}
\end{align*}

Note: One can find internal stress via numerical differentiation as in FE/FD
but these are not as accurate as the above expressions.

Expressions for the new tensors $ u_{ijk}^{*}$ and $t_{ijk}^{*}$ are on page
191 in \cite{brebbia:1984}.

\section{Boundary Integral Equation}
\label{sec:BIE,sec4.10}

Just as we did for Laplace's equation we need to consider the limiting case of
\eqnref{eqn:eqnF} as $P$ is moved to $\del \Omega$. (\ie we need to find the
equivalent of $\fnof{c}{P}$ (in section 3) - called here $\fnof{c_{ij}}{P}$.)
We use the same procedure as for Laplace's equation but here things are not so
easy.

If  $P \in \del \Omega$ we enlarge $\Omega$ to $\Omega^{\prime}$ as 
shown.

\begin{figure}[htbp] \centering
  \input{figs/lin_elasticity/illus.pstex}
  \caption{Illustration of enlarged domain when singular point is on the 
    boundary.}
\end{figure}
Then \eqnref{eqn:eqnF} can be written as
\begin{multline}
  \fnof{u_{i}}{P} = \gint{\Gamma_{-\varepsilon} +\Gamma_{\varepsilon}}{}
  {\fnof{u_{ij}^{*}}{P,\vect{x}}
    \fnof{t_{j}}{\vect{x}}}{\fnof{\Gamma}{\vect{x}}} -
  \gint{\Gamma_{-\varepsilon}+
    \Gamma_{\varepsilon}}{}{\fnof{t_{ij}^{*}}{P,\vect{x}}
    \fnof{u_{j}}{\vect{x}}}{\fnof{\Gamma}{\vect{x}}} \\ +
    \gint{\Omega^{\prime}}{}{\fnof{u_{ij}^*}{P,\vect{x}} \fnof{b_{j}}{\vect{x}}}
    {\fnof{\Omega}{\vect{x}}} 
  \label{eqn:Fwrit}
\end{multline} 

We need to look at each integral in turn as $\varepsilon^{\downarrow}0$ (\ie
$\varepsilon \rightarrow 0$ from above).  The only integral that presents a
problem is the second integral. This can be written as
\begin{multline}
  \gint{\Gamma_{-\varepsilon}+\Gamma_{\varepsilon}}{}{\fnof{t_{ij}^{*}}{P,\vect{x}} \fnof{u_{j}}{\vect{x}}}{\fnof{\Gamma}{\vect{x}}} =
  \gint{\Gamma_{\varepsilon}}{}{\fnof{t_{ij}^{*}}{P,\vect{x}}
    \fnof{u_{j}}{\vect{x}}}{\fnof{\Gamma}{\vect{x}}} \\ +
  \gint{\Gamma_{-\varepsilon}}{}{\fnof{t_{ij}^{*}}{P,\vect{x}}
    \fnof{u_{j}}{\vect{x}}}{\fnof{\Gamma}{\vect{x}}}
  \label{eqn:2ndint}
\end{multline} 

The first integral on the right hand side can be written as
\begin{multline}
  \gint{\Gamma_{\varepsilon}}{}{\fnof{t_{ij}^{*}}{P,\vect{x}}
    \fnof{u_{j}}{\vect{x}}}{\fnof{\Gamma}{\vect{x}}} =
  \underbrace{\gint{\Gamma_{\varepsilon}}{}{\fnof{t_{ij}^{*}}{P,\vect{x}}
      \sqbrac{\fnof{u_{j}}{\vect{x}}-\fnof{u_{j}}{P}}}{\Gamma(x)}}_{\text{$0$
      by continuity of $\fnof{u_{j}}{x}$}} \\ + \fnof{u_{j}}{P}
  \gint{\Gamma_{\varepsilon}}{}{\fnof{t_{ij}^{*}}{P,\vect{x}}}
  {\fnof{\Gamma}{\vect{x}}}
\label{eqn:1stint}
\end{multline}  
  
Let
\begin{equation}
  \fnof{c_{ij}}{P} = \delta_{ij} + \lim_{\varepsilon \downarrow 0}
   \gint{\Gamma_{\varepsilon}}{}{\fnof{t_{ij}^{*}}{P,\vect{x}}}
   {\fnof{\Gamma}{\vect{x}}}
 \label{eqn:eqnG}
\end{equation}

As $\varepsilon^{\downarrow} 0$, $\Gamma_{-\varepsilon}\rightarrow \Gamma$ and
we write the second integral of \eqnref{eqn:2ndint} as $\gint{\Gamma}{}
{\fnof{t_{ij}^{*}}{P,\vect{x}}
  \fnof{u_{j}}{\vect{x}}}{\fnof{\Gamma}{\vect{x}}}$ where we interpret this in
the Cauchy Principal Value\footnote{What is a Cauchy Principle Value? 
                \newline Consider $\fnof{f}{x} = \dfrac{1}{x}$ on
    $\Gamma_{-\varepsilon} = \sqbrac{-2,-\varepsilon) \cup (\varepsilon,2}$
   \newline Then
      \begin{alignat*}{2}
        \gint{\Gamma_{-\varepsilon}}{}{\fnof{f}{x}}{x} &=
        \gint{-2}{-\varepsilon}{\dfrac{1}{x}}{x} +
        \gint{\varepsilon}{2}{\dfrac{1}{x}}{x} 
        = \evalat{\ln \abs{x}}{{-2}}^{-\varepsilon} + 
          \evalat{\ln\abs{x}}{{\varepsilon}}^{2} && \\ 
          &= \ln \varepsilon -\ln 2 + \ln 2
        - \ln \varepsilon=0 \;\;\forall \varepsilon>0 && \\ & \Rightarrow
        \displaystyle{\lim_{\varepsilon \rightarrow 0}}
        \gint{\Gamma_{-\varepsilon}}{}{\fnof{f}{x}}{x} = 0 && 
                                \text{This is the Cauchy Principle Value of}
                        \gint{\Gamma}{}{\fnof{f}{x}}{x}
%%                                \label{eqn:CPV}
      \end{alignat*}
   But if we replace $\Gamma_{-\varepsilon}$  by 
   $\displaystyle{\lim_{\varepsilon\rightarrow 0}}\Gamma_{-\varepsilon} = 
   \sqbrac{-2,2}  = \Gamma$ then 
   \begin{align*}
        \gint{\Gamma}{}{\dfrac{1}{x}}{x} = \gint{-2}{2}
        {\dfrac{1}{x}}{x} = \pbrac{\displaystyle{\lim_{\varepsilon_{1} 
              \rightarrow 0}}\gint{-2}{\varepsilon_{1}}{\dfrac{1}{x}}{x}}  +
        \pbrac{\displaystyle{\lim_{\varepsilon_{2} 
              \rightarrow 0}}\gint{-\varepsilon_{2}}{2}{\dfrac{1}{x}}{x}}
        &\text{(by definition of improper integration)}
%        \label{eqn:bydefn}
    \end{align*}
    \newline which does NOT exist. \ie the integral does not exist in the
    proper sense, but it does in the Cauchy Principal Value sense. However, if
    an integral exists in the proper sense, then it exists in the Cauchy
    Principal Value sense and the two values are the same.}sense. 

Thus as $\varepsilon^{\downarrow}0$ we get the boundary integral equation
\begin{multline}
  \fnof{c_{ij}}{P} \fnof{u_{j}}{P} +
  \gint{\Gamma}{}{\fnof{t_{ij}^{*}}{P,\vect{x}}
    \fnof{u_{j}}{\vect{x}}}{\fnof{\Gamma}{\vect{x}}} \\
  = \gint{\Gamma}{}{\fnof{u_{ij}^{*}}{P,\vect{x}}
    \fnof{t_{j}}{\vect{x}}}{\fnof{\Gamma}{\vect{x}}} 
  + \goneint{\fnof{u_{ij}^{*}}{P,\vect{x}} \fnof{b_{j}}{\vect{x}}}{\Omega}
  \label{eqn:eqnH}
\end{multline}
(or, in brief (no body force), $c_{ij} u_{j} + \gint{\Gamma}{}{t_{ij}^{*}
u_{j}}{\Gamma} = \gint{\Gamma}{}{u_{ij}^{*} t_{j}}{\Gamma} $) where the
integral on the left hand side is interpreted in the Cauchy Principal sense.
In practical applications $c_{ij}$ and the principal value integral can be
found indirectly from using \eqnref{eqn:eqnH} to represent rigid-body
movements.

The numerical implementation of \eqnref{eqn:eqnH} is similar to the numerical 
implementation of an elliptic equation (\eg Laplace's Equation).  However, whereas
with Laplace's equation the unknowns were $u$ and $\delby{u}{n}$ (scalar 
quantities) here the unknowns are vector quantities. Thus it is more 
convenient to work with matrices instead of indicial notation.
\newline \ie use
\begin{displaymath}
  \vect{u} = \begin{bmatrix}
    u_{1} \\
    u_{2} \\
    u_{3}
  \end{bmatrix}, \qquad 
  \vect{t} = \begin{bmatrix}
    t_{1} \\ 
    t_{2} \\
    t_{3}
  \end{bmatrix}
\end{displaymath}
\begin{displaymath}
  \vect{u}^{*} = \begin{bmatrix}
    u_{11}^{*} &  u_{12}^{*} &  u_{13}^{*}\\
    u_{21}^{*} &  u_{22}^{*} &  u_{23}^{*}\\
    u_{31}^{*} &  u_{32}^{*} &  u_{33}^{*}
  \end{bmatrix}, \qquad
  \vect{t}^{*} = \begin{bmatrix}
    t_{11}^{*} &  t_{12}^{*} &  t_{13}^{*}\\
    t_{21}^{*} &  t_{22}^{*} &  t_{23}^{*}\\
    t_{31}^{*} &  t_{32}^{*} &  t_{33}^{*}
  \end{bmatrix}
\end{displaymath}
Then (in absence of a body force) we can write \eqnref{eqn:eqnH} as 
\begin{equation}
  \vect{cu} + \goneint{\vect{t}^{*}\vect{u}}{\Gamma} =
  \goneint{\vect{u}^{*}\vect{t}}{\Gamma}
  \label{eqn:eqnH*}
\end{equation}

We can discretise the boundary as before and put $P$, the singular point, at
each node (each node has $6$ unknowns - $3$ displacements and $3$ tractions - we get
$3$ equations per node). The overall matrix equation
\begin{equation}
  \matr{A}\vect{U}=\matr{B}\vect{T}
  \label{eqn:matrix}
\end{equation}
where $\vect{U} = 
\begin{bmatrix}
  \vect{u}_{1} \\ 
  \vect{u}_{2} \\ 
  \vdots \\ 
  \vect{u}_{n}
\end{bmatrix} $ and  $ \vect{t} = 
\begin{bmatrix}
  \vect{t}_{1} \\
  \vect{t}_{2} \\ 
  \vdots \\
  \vect{t}_{n}
\end{bmatrix}$ where $n$ is the number nodes.

The diagonal elements of the $\matr{A}$ matrix in \eqnref{eqn:matrix} 
(for three-dimensions, a $3$ x $3$
matrix) contains principal value components. If we have a rigid-body
displacement of a \emph{finite} body in any one direction then we get
\begin{equation*}
  \matr{A}\vect{I}_{l} = \vect{0}
\end{equation*}
($\vect{I}_{l}$ = vector defining a rigid body displacement in direction $l$)
\begin{displaymath}
    \Rightarrow a_{ii} = -\dsuml{i\neq j}{}a_{ij} \qquad \text{(no sum on $i$)} 
\end{displaymath} 
\ie the diagonal entries of $\matr{A}$ (the $c_{ij}$'s) do not need to be
determined explicitly. There is a similar result for an infinite body.

\section{Body Forces (and Domain Integrals in General)}

The body force gives rise to a domain integral although it does not give rise
to any further unknowns in the system of equations.  (This is because the body
force is known - the fundamental solution was chosen so that it removed all
unknowns appearing in domain integrals).

Thus \eqnref{eqn:eqnH} is still classed as a Boundary Integral Equation.
Integrals over the domain containing known functions (eg body force integral)
appear in many situations \eg the Poisson equation $\laplacian{u}=f$ yields a
domain integral involving $f$.

The question is how do we evaluate domain integrals such as those appearing in
the boundary integral forumalation of such equations?  Since the functions are
known a \emph{coarse} domain mesh may work.(\nb Since the integral also
contains the fundamental solution and $\Omega$ may not be a ``nice'' region it
is unlikely that it can be evaluated analytically). However, a domain mesh
nullifies one of the advantages of BEM - that of having to prepare only a
boundary mesh.

In some cases domain integrals must be used but there are techniques
developing to avoid many of them.  In some standard situations a domain
integral can be transformed to a boundary integral. \eg a body force arising
from a constant gravitational load, or a centrifugal load due to rotation
about a fixed axis or the effect of a steady state thermal load can all be
transformed to a boundary integral.

Firstly, let $G_{ij}^{*}$ (the Galerkin tension) be related to $u_{ij}^{*}$ by
\begin{align*}
    u_{ij}^{*} &= G_{ij,kk}^{*} -  \dfrac{1}{2\pbrac{1-\nu}}G_{ik,kj}^{*} \\
    \Rightarrow G_{ij} &=
    \left\{ \begin{array}{ll}
        \dfrac{1}{8 \pi G} r \delta_{ij} & \text{ (3D)}\\
        \dfrac{1}{8 \pi G}  r^{2} \log \pbrac{\dfrac{1}{r}} \delta_{ij}
        & \text{ (2D)}
      \end{array} \right.
\end{align*}

Then
\begin{displaymath}
  B_{i}  =  \goneint{u_{ij}^{*} b_{j}}{\Omega} =   
  \goneint{\pbrac{G_{ij,kk}^{*} -\dfrac{1}{2\pbrac{1-\nu}} G_{ik,kj}^{*}}
  b_{j}}{\Omega}
\end{displaymath} 

Under a constant gravitational load $\vect{g}=\pbrac{g_{j}}$
\begin{align*}
  b_{j} &= \rho g_{j}\\
  \Rightarrow B_{i} &= \rho g_{j} \goneint{\pbrac{G_{ik,j}^{*} 
  -\dfrac{1}{2\pbrac{1-\nu}} G_{ik,kj}^{*}}}{\Omega} \\
  &= \rho g_{j} \goneint{\bbrac{G_{ij,k}^{*} -\dfrac{1}{2\pbrac{1-\nu\}} 
  G_{ik,j}^{*}}}n_{k}}{\Gamma}
\end{align*}
which is a boundary integral. 

Unless the domain integrand is ``nice'' the above simple application of
Green's theorem won't work in general. There has been a considerable amount of
research on domain integrals in BEM which has produced techniques for
overcoming some domain methods. The two integrals of note are the DRM, dual
reciprocity method, developed around 1982 and the MRM, multiple reciprocity
method, developed around 1988.

\clearpage 
\section{CMISS Examples}

\begin{enumerate}
\item   To solve a truss system run CMISS example $411$
  This solves the simple three truss system shown in
  \Figref{fig:system}. 
  \label{xmp:Solving}

\item  To solve stresses in a bicycle frame modelled with truss elements 
  run CMISS example $412$.
  \begin{figure}[htbp] \centering
    \input{figs/lin_elasticity/CMISSA.pstex}
  \end{figure}
  \label{xmp:stressesinabike}
\end{enumerate}

%\begin{example}{Stresses in a bicycle frame modelled with beam elements}
%  {Stresses in a bicycle frame modelled with beam elements.}
%  \todo{example???}

%  \begin{figure}[htbp] \centering
%    \input{figs/CMISSA.pstex}
%  \end{figure}
%  \label{xmp:Sbfbe}
%\end{example}
%\index{Beam elements|)}

%\begin{example}{Stresses in a plate with a hole}
%  {Stresses in a plate with a hole}
  
%  \todo{example???}

%  A common problem in structural mechanics is to find the stress concentration
%  produced by a hole in an otherwise uniformly loaded structural component.
%  Consider the plate below loaded by horizontal forces of \nunit{100}{\kNpm}.
%  The plate is \nunit{1}{\m} thick and made of steel (Young's modulus
%  \nunit{100}{\GPa}, Poisson's ratio $0.3$).

%  \begin{figure}[htbp] \centering
%    \input{figs/lin_elasticity/CMISSB.pstex}
%  \end{figure}
%  \label{xmp:stressesinplate}
%\end{example}


%%% Local Variables: 
%%% mode: latex
%%% TeX-master: "/product/cmiss/documents/notes/fembemnotes/fembemnotes"
%%% End: 

